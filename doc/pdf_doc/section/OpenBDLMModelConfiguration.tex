\subsection{Model configuration}
\label{S:MODELCONFIGURATION}
%After loading and pre-processing the data, the next step of the analysis is the model configuration.
Model configuration includes, (i) defining the number of model class, (ii) defining the dependencies between the time series in case of multiple time series analysis, (iii) defining the block components which are assigned to each time series and model class, (iv) defining possible constrain between model parameters.
%The dependencies between time series as well as constrains between model parameters of different model class are also handled during the configuration of the model.
The \MATLAB{} structure named \lstinline[basicstyle = \mlttfamily]!model.component! stores all the information about the model configuration.
The structure \lstinline[basicstyle = \mlttfamily]!model.component! has three fields named  \lstinline[basicstyle = \mlttfamily]!model.component.ic!, \lstinline[basicstyle = \mlttfamily]!model.component.block!, and \lstinline[basicstyle = \mlttfamily]!model.component.const!.

\subsubsection{Dependencies between time series}

The field \lstinline[basicstyle = \mlttfamily]!model.component.ic! stores the information related to time series dependencies.
\lstinline[basicstyle = \mlttfamily \small ]!model.component.ic! stores a $1\times \mathtt{D}$ cell array of $(1\times \mathtt{D}-1)$ matrix, where $\mathtt{D}$ is the number of time series.
Each time series depend on the time series corresponding to the indexes given in the $\mathtt{D}$ arrays.
If the array is empty, the time series are independent.

\subsubsection{Block components}

OpenBDLM supports 4 types of block components: (1) baseline, (2) periodic, (3) periodic kernel regression, and (4) autoregressive 
(1) The baseline component models the local mean of the time series. 
There are three types of baseline supported in OpenBDLM: (i) level model, (ii) trend model, (iii) acceleration model. 
(2) The periodic component models harmonic periodic phenomena. %, which are most often related to external effects.
(3) The periodic kernel regression models non-harmonic periodic pattern.
(4) The autoregressive component is intended as a residual term to capture the time dependent model errors.
The field \lstinline[basicstyle = \mlttfamily \small ]!block! stores a $1\times \mathtt{S}$ cell array, where $\mathtt{S} = \{1,2 \}$ is the number of model classes.
Each cell array is a $1\times \mathtt{D}$ cell array of array, where $\mathtt{D}$ is the number of time series.
Each block component is associated with a reference number:
\begin{itemize}
\item 11: Local level 
\item 12: Local trend
\item 13: Local acceleration
\item 21: Local level compatible with local trend
\item 22: Local level compatible with local acceleration
\item 23: Local trend compatible with local acceleration
\item 31: Periodic
\item 41: First-order autoregressive
\item 51: Kernel regression
\end{itemize}
The number of hidden states associated with each block component can be different. 
Each block component can be replicated, each having its own set of model parameters. 
For instance, two periodic components with periods of 365 days and 1 day can be used to model seasonal and daily variations, respectively.

\subsubsection{Parameter constrains}

The variable \lstinline[basicstyle = \mlttfamily]!model.component.const! stores a $1\times \mathtt{S}$ cell array, where $\mathtt{S} = \{1, 2 \}$ is the total number of model classes.
It is defined only if $\mathtt{S} = 2$.
The first cell is empty, and the second cell is a $1\times \mathtt{D}$ cell array of array, where $\mathtt{D}$ is the number of time series.
The array contains $0$ and $1$ to indicate which block components of the second model class has the same model parameters than the corresponding component of the first model class. 
A value of $1$ indicates that the model parameters are constrained between the block components of the two model classes, $0$ otherwise.

\subsubsection{Number of model class}

OpenBDLM is capable of detecting regime changes in the dynamics in the baseline of the time series. Therefore, OpenBDLM handles model switching between the three types of baseline dynamics, that is, local level, local trend, and local acceleration models. OpenBDLM supports a maximum of two model dynamics, which includes six types of model switch: (1) from local level model to local trend model (and reverse), (2) from local level model to acceleration model (and reverse), (3) from local trend to acceleration model (and reverse).


\subsubsection{Model configuration functions}

The model configuration workflow is presented Figure~\ref{FIG:ModelConfigurationWorkflow}. The list of OpenBDLM functions used for model configuration is:

\begin{description}[style=unboxed]
\item[Controls script for model configuration ] \leavevmode
  \begin{lstlisting}[ basicstyle = \mlttfamily \small, breaklines=true]
[data,model,estimation,misc]=ModelConfiguration(data,model,estimation,misc)
 \end{lstlisting}
 
 \item[Model configuration for real data ] \leavevmode
  \begin{lstlisting}[ basicstyle = \mlttfamily \small, breaklines=true]
[data,model,estimation,misc]=configureModelForDataReal(data,model,estimation,misc)
 \end{lstlisting}
 
  \item[Model configuration for synthetic data (for synthetic data creation only)] \leavevmode
  \begin{lstlisting}[ basicstyle = \mlttfamily \small, breaklines=true]
[data,model,estimation,misc]=configureModelForDataSimulation(data,model,estimation,misc)
 \end{lstlisting}
 
   \item[Requests user's input to configure the model] \leavevmode
  \begin{lstlisting}[ basicstyle = \mlttfamily \small, breaklines=true]
[model,misc]=defineModel(data,misc)
 \end{lstlisting}
 
    \item[Requests user's input to define time series labels/reference name (for synthetic data creation only)] \leavevmode
  \begin{lstlisting}[ basicstyle = \mlttfamily \small, breaklines=true]
  [data,misc]=defineDataLabels(data,misc)
 \end{lstlisting}
 
     \item[Requests user's input to define data timestamps (for synthetic data creation only)] \leavevmode
  \begin{lstlisting}[ basicstyle = \mlttfamily \small, breaklines=true]
[data,misc]=defineTimestamps(data,misc)
 \end{lstlisting}
 
 
\end{description}


\begin{figure}[!h]
  \centering
  \captionsetup{justification=centering}
\scalebox{0.8}{
\begin{tikzpicture}

\node[parabisque](inputModelConfiguration){\begin{tabular}{c} \lstinline[basicstyle = \mlttfamily \small ]!data! \\ \lstinline[basicstyle = \mlttfamily \small ]!model! \end{tabular}};
\node[esbisque][below of = inputModelConfiguration, yshift = -1cm ](ModelConfiguration){\phantom{} ModelConfiguration.m \phantom{}};
\node[testbisque](testSimulation)[below of = ModelConfiguration, yshift = -2cm]{\begin{tabular}{c} synthetic \\ data ?  \end{tabular}};
\node[esbisque](ModelConfigurationReal)[below of = testSimulation, yshift = -1cm, xshift = -3.5cm]{\phantom{} configureModelForDataReal.m \phantom{}};
\node[esbisque](ModelConfigurationSimulation)[below of = testSimulation, yshift = -1cm, xshift = 4cm]{\phantom{} configureModelForDataSimulation.m \phantom{}};
\node[esbisque](defineDataLabels)[below of = ModelConfigurationSimulation, yshift = -1cm]{\phantom{} defineDataLabels.m \phantom{}};
\node[esbisque](defineTimestamps)[below of = defineDataLabels, yshift = -1cm]{\phantom{} defineTimestamps.m \phantom{}};
\node[esbisque](defineModel)[below of = ModelConfiguration, yshift = -10cm]{\phantom{} defineModel.m \phantom{}};
\node[parabisque](outputModelConfig)[below of = defineModel, yshift = -1cm]{\begin{tabular}{c} \lstinline[basicstyle = \mlttfamily \small ]!model.components.block! \\ \lstinline[basicstyle = \mlttfamily \small ]!model.components.ic! \\  \lstinline[basicstyle = \mlttfamily \small ]!model.components.const!  \end{tabular}};

\path[->, thick] (inputModelConfiguration)edge(ModelConfiguration);
\path[->, thick] (ModelConfiguration)edge(testSimulation);
\path[->, draw, thick] (testSimulation.west) -| (-2cm,-5cm) -| node[pos=0.25, rotate=0, fill=white]{ no} (ModelConfigurationReal.north);
\path[->, draw, thick] (testSimulation.east) -| (2cm,-5cm) -| node[pos=0.25, rotate=0, fill=white]{ yes} (ModelConfigurationSimulation.north);
\path[->, thick] (ModelConfigurationSimulation)edge(defineDataLabels);
\path[->, thick] (defineDataLabels)edge(defineTimestamps);
\path[->, draw, thick] (defineTimestamps.south) |- (4cm,-12cm) -|  (defineModel.north);
\path[->, draw, thick] (ModelConfigurationReal.south) |- (-3cm,-12cm) -|  (defineModel.north);
\path[->, thick] (defineModel)edge(outputModelConfig);

\end{tikzpicture} } 
\caption{Model configuration workflow} \label{FIG:ModelConfigurationWorkflow}
\end{figure}