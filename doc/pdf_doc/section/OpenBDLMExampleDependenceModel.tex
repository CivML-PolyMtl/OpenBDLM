\subsection{Example \#2: dependence model between two time series}
\label{S:ExampleDispTemp}

\subsubsection{Data description}

This example uses two synthetic time series that mimics the displacement and temperature data measured on a bridge.
The  Figure~\ref{fig:DataSummaryRaw2}a shows that data points exist between August 2013 and October 2015.
The timestep in the original data is non-uniform; it varies from 1 hour to 25 hours (see Figure~\ref{fig:DataSummaryRaw2}b). 
The timestep vector is not identical on each time series. 
It means that the time series are not synchronized between each other.
The most frequent (i.e referent) time step is 1 hour for both time series (Section~\ref{SS:NonUniform}).
There is no missing data (\lstinline[basicstyle = \mlttfamily \small, backgroundcolor = \color{light-gray}]!NaN!) on the displacement time series, but there are missing data on the temperature time series as indicated by the red crosses on the Figure~\ref{fig:DataSummaryRaw2}c.
Each red cross indicates the presence of a Not a Number (\lstinline[basicstyle = \mlttfamily \small, backgroundcolor = \color{light-gray}]!NaN!) value in the time series.
After data synchronization, the time step vectors are identical on each time series (Figure~\ref{fig:DataSummaryDefaultPreProcessed2}).
Both time series are stationary, and they exhibit a level, a yearly and daily periodic pattern as well as an autoregressive pattern.
The periodic patterns observed on the displacement time series is due to the temperature variations observed in the temperature time series.

In this example, we choose to resample the original data in order to have timesteps of 6h instead of 1h. 

\begin{figure*}[h!]
\centering
\begin{subfigure}{\linewidth}
\centering
\includegraphics[width=0.9\linewidth]{./docfigs/Example_DISPTEMPSIM/raw/ALL_AMPLITUDES.pdf} 
\caption{Amplitude}
\end{subfigure}
\begin{subfigure}{\linewidth}
\centering
\includegraphics[width=0.9\linewidth]{./docfigs/Example_DISPTEMPSIM/raw/ALL_TIMESTEPS.pdf}
\caption{Timestep}
\end{subfigure}
\begin{subfigure}{\linewidth}
\centering
\includegraphics[width=0.9\linewidth]{./docfigs/Example_DISPTEMPSIM/raw/AVAILABILITY.pdf}
\caption{Availability}
\end{subfigure}
\caption{Raw data in the example \#2 where the reference timestep is 1h.}
\label{fig:DataSummaryRaw2}
\end{figure*}


\begin{figure*}[h!]
\centering
\begin{subfigure}{\linewidth}
\centering
\includegraphics[width=0.9\linewidth]{./docfigs/Example_DISPTEMPSIM/preprocessed_default/ALL_AMPLITUDES.pdf} 
\caption{Amplitude}
\end{subfigure}
\begin{subfigure}{\linewidth}
\centering
\includegraphics[width=0.9\linewidth]{./docfigs/Example_DISPTEMPSIM/preprocessed_default/ALL_TIMESTEPS.pdf}
\caption{Timestep}
\end{subfigure}
\begin{subfigure}{\linewidth}
\centering
\includegraphics[width=0.9\linewidth]{./docfigs/Example_DISPTEMPSIM/preprocessed_default/AVAILABILITY.pdf}
\caption{Availability}
\end{subfigure}
\caption{Data used in example \#2 after resampling to obtain a reference timestep of 6h.}
\label{fig:DataSummaryDefaultPreProcessed2}
\end{figure*}



\subsubsection{Model description}
\label{SS:ModelConstructionExample2}

The model includes one model class, and the hidden state variables are 
\begin{gather*}
\textbf{x}=[x^{\mathtt{LL}}_{\mathtt{D}}, x^{\mathtt{AR}}_{\mathtt{D}}, x^{\mathtt{LL}}_{\mathtt{T}}, x^{\mathtt{P1}\text{,yearly}}_{\mathtt{T}}, x^{\mathtt{P2}\text{,yearly}}_{\mathtt{T}}, x^{\mathtt{P1}\text{,daily}}_{\mathtt{T}} , x^{\mathtt{P2}\text{,daily}}_{T}, x^{\mathtt{AR}}_{\mathtt{T}}],
\end{gather*}
where $\mathtt{D}$ and $\mathtt{T}$ refer to the displacement and temperature time series, respectively.
The periodic patterns observed on the displacement are considered through a dependency of the displacement on the hidden state variables of the periodic  and autoregressive components of the temperature time series (Section~\ref{S:Dependencies}).
The associated model parameters are
\begin{align*}
\bm\theta & =[\sigma_{w, \mathtt{D}}^{\mathtt{LL}}, \phi^{\mathtt{AR}}_{D}, \sigma_{w, \mathtt{D}}^{\mathtt{AR}}, \sigma_{v, \mathtt{D}},  \\
&  \sigma_{w, \mathtt{T}}^{\mathtt{LL}},  p^{\mathtt{P}, \text{yearly}}_{\mathtt{T}}, \sigma_{w, \mathtt{T}}^{\mathtt{P}, \text{yearly}} , p^{\mathtt{P}, \text{daily}}_{\mathtt{T}} , \sigma_{w, \mathtt{T}}^{\mathtt{P}, \text{daily}}, \phi^{\mathtt{AR}}_{\mathtt{T}}, \sigma_{w, \mathtt{T}}^{\mathtt{AR}}, \sigma_{v, \mathtt{T}},\phi^{\mathtt{D}|\mathtt{T}}_{\mathtt{P}_{y}},\phi^{\mathtt{D}|\mathtt{T}}_{\mathtt{P}_{d}},\phi^{\mathtt{D}|\mathtt{T}}_{\mathtt{AR}}].
\end{align*}
The optimized model parameters values computed using the Newton-Raphson algorithm (see~\ref{SS:THModelParameterEstimation}) with a training period of 180 days are
\begin{align*}
 \bm\theta^{\text{*}}& =[0, 0.90, 0.037, 1.94\times10^{-5},  \\
 & 0, 365.2422, 0, 1, 0, 0.98, 0.86, 1.14\times10^{-4}, -0.013, 0.0706, 0.00073 ].
\end{align*}
The estimated initial hidden states mean and covariance values are 
\begin{align*}
\bm \mu^{*}_{0} & = [	25.9,-2.55\times10^{-5},5.53,16.3,-0.999,0.263,0.669,3.77 ]^{\intercal}, \text{and} \\
\bm\Sigma^{*}_{0} & = \text{diag}([4.27\times10^{-5},1.68\times10^{-3},0.715,0.338,0.341,6.73\times10^{-5},6.73\times10^{-5},1.81]).
 \end{align*}
The hidden states computed using the estimated model parameters and initial hidden states are presented in Figure~\ref{fig:DISPTEMPSIMOptimizedOptimizedExample2}.


\subsubsection{Run the example from the pre-existing configuration file}
\label{SS:LoadConfigFileEx2}
There is a configuration file CFG\_Example\_DISPTEMP\_optim.m which is located in the ``config\_files'' folder of the OpenBDLM package.
CFG\_Example\_DISPTEMP\_optim.m contains the optimized model parameters and optimized initial hidden states values.
There is also a data file DATA\_Example\_DISPTEMP\_optim.mat that is located in the ``data/mat'' subfolder.
Therefore, it is possible to run the example \#2 by following the steps below while interacting with the \MATLAB{} command line:
\begin{enumerate}
\item Start OpenBDLM. Type \colorbox{light-gray}{\lstinline[basicstyle = \mlttfamily \small, backgroundcolor = \color{light-gray}]!OpenBDLM_main('CFG_Example_DISPTEMP_optim.m');!}.
\item Access hidden states estimation menu. Type \colorbox{light-gray}{\lstinline[basicstyle = \mlttfamily \small, backgroundcolor = \color{light-gray}]!3!}.
\item Run the Kalman filter to estimate the hidden states. Type \colorbox{light-gray}{\lstinline[basicstyle = \mlttfamily \small, backgroundcolor = \color{light-gray}]!1!}.
\item Save and quit. Type \colorbox{light-gray}{\lstinline[basicstyle = \mlttfamily \small, backgroundcolor = \color{light-gray}]!Q!}.
\end{enumerate}


\subsubsection{Run the example from command line interaction}

The analysis of a new dataset usually requires to start from scratch.
This section explains how to run the example \#2 from scratch, that is, how to load  and resample the data presented in Figure~\ref{fig:DataSummaryRaw2}, configure the model, estimate the model parameters and estimate the hidden states.
This may be done by following steps below while interacting with the \MATLAB{} command line:
\begin{enumerate}
\item Start OpenBDLM. Type \colorbox{light-gray}{\lstinline[basicstyle = \mlttfamily \small, backgroundcolor = \color{light-gray}]!OpenBDLM_main;!}.
\item Choose the interactive tool. Type \colorbox{light-gray}{\lstinline[basicstyle = \mlttfamily \small, backgroundcolor = \color{light-gray}]!0!}.
\item Enter the project name. Type \colorbox{light-gray}{\lstinline[basicstyle = \mlttfamily \small, backgroundcolor = \color{light-gray}]!Example_DISPTEMP!}. 
\item Disregard generating synthetic data. Type \colorbox{light-gray}{\lstinline[basicstyle = \mlttfamily \small, backgroundcolor = \color{light-gray}]!no!}. 
\item Load new data. Type \colorbox{light-gray}{\lstinline[basicstyle = \mlttfamily \small, backgroundcolor = \color{light-gray}]!0!}.
\item Select from the graphical user interface the data files located in the ``/data/csv/Example\_DISPTEMP/'' folder. The Figure~\ref{fig:DataSummaryRaw2} that represents the raw data should popup on screen.
\item Access the resampling menu. Type \colorbox{light-gray}{\lstinline[basicstyle = \mlttfamily \small, backgroundcolor = \color{light-gray}]!4!}. 

\item Resample data to obtain timesteps of 6h (0.25 day). Type \colorbox{light-gray}{\lstinline[basicstyle = \mlttfamily \small, backgroundcolor = \color{light-gray}]!0.25!}. The Figure \ref{fig:DataSummaryDefaultPreProcessed2} should popup on the screen this time for the resampled data.

\item Save and continue. Type \colorbox{light-gray}{\lstinline[basicstyle = \mlttfamily \small, backgroundcolor = \color{light-gray}]!7!}. The same figures should popup on the screen again.
\item Select dependency for the time series \#1. Type \colorbox{light-gray}{\lstinline[basicstyle = \mlttfamily \small, backgroundcolor = \color{light-gray}]![2]!}.
\item Select dependency for the time series \#2. Type \colorbox{light-gray}{\lstinline[basicstyle = \mlttfamily \small, backgroundcolor = \color{light-gray}]![0]!}.
\item Select the number of model classes. Type \colorbox{light-gray}{\lstinline[basicstyle = \mlttfamily \small, backgroundcolor = \color{light-gray}]!1!}. 
\item Select the model block components for time series \#1. Type \colorbox{light-gray}{\lstinline[basicstyle = \mlttfamily \small, backgroundcolor = \color{light-gray}]![11 41]!}.
\item Select the model block components for time series \#2. Type \colorbox{light-gray}{\lstinline[basicstyle = \mlttfamily \small, backgroundcolor = \color{light-gray}]![11 31 31 41]!}.

\item Access the training period modification menu. Type \colorbox{light-gray}{\lstinline[basicstyle = \mlttfamily \small, backgroundcolor = \color{light-gray}]!13!}. 

\item Modify the training period. Type \colorbox{light-gray}{\lstinline[basicstyle = \mlttfamily \small, backgroundcolor = \color{light-gray}]!1!}. 

\item Choose the starting time (day). Type \colorbox{light-gray}{\lstinline[basicstyle = \mlttfamily \small, backgroundcolor = \color{light-gray}]!1!}. 

\item Choose the end time (day). Type \colorbox{light-gray}{\lstinline[basicstyle = \mlttfamily \small, backgroundcolor = \color{light-gray}]!180!}. 

%\item Access hidden states estimation menu. Type \colorbox{light-gray}{\lstinline[basicstyle = \mlttfamily \small, backgroundcolor = \color{light-gray}]!3!}. 
%\item Change the state estimation method to UD\footnote{The UD computation is required in this case because of the presence of missing data. See Section~\ref{SS:KFUD} for more details.}. Type \colorbox{light-gray}{\lstinline[basicstyle = \mlttfamily \small, backgroundcolor = \color{light-gray}]!3!}. 
%\item Return to the main menu. Type \colorbox{light-gray}{\lstinline[basicstyle = \mlttfamily \small, backgroundcolor = \color{light-gray}]!R!}.
\item Access model parameter estimation menu. Type \colorbox{light-gray}{\lstinline[basicstyle = \mlttfamily \small, backgroundcolor = \color{light-gray}]!1!}. 
\item Start the Newton-Raphson algorithm. Type \colorbox{light-gray}{\lstinline[basicstyle = \mlttfamily \small, backgroundcolor = \color{light-gray}]!1!}. Once the algorithm has converged, the optimized model parameters values should be close to the values presented in \S\ref{SS:ModelConstructionExample2}. Note also that it is possible to get slightly different parameter values\footnote{Keep in mind that the optimization may take several minutes. It is possible to abort the analysis here and to load the configuration file called CFG\_Example\_DISPTEMP\_optim.m to load pre-computed values of model parameters, as presented in Section~\ref{SS:LoadConfigFileEx2}.}.
\item Estimate the initial hidden states values. Type \colorbox{light-gray}{\lstinline[basicstyle = \mlttfamily \small, backgroundcolor = \color{light-gray}]!2!}.
\item Estimate the hidden states using the Kalman filter. Type \colorbox{light-gray}{\lstinline[basicstyle = \mlttfamily \small, backgroundcolor = \color{light-gray}]!1!}. The estimation should be similar to the results presented in Figure~\ref{fig:DISPTEMPSIMOptimizedOptimizedExample2}.
\item Access export menu. Type \colorbox{light-gray}{\lstinline[basicstyle = \mlttfamily \small, backgroundcolor = \color{light-gray}]!17!}. 
\item Export the current project in a configuration file. Type \colorbox{light-gray}{\lstinline[basicstyle = \mlttfamily \small, backgroundcolor = \color{light-gray}]!1!}.
\item Save and quit OpenBDLM. Type \colorbox{light-gray}{\lstinline[basicstyle = \mlttfamily \small, backgroundcolor = \color{light-gray}]!Q!}.
\end{enumerate}




%\subsection{Step 4: configure the model}
%
%The next step is to configure the model.
%First, the program requests the number of model class.
%In this exemple, the time series data looks stationary and we are not interested in anomaly detection, and therefore we type \colorbox{light-gray}{\lstinline[basicstyle = \mlttfamily \small, backgroundcolor = \color{light-gray}]!1!}.
%Secondly, because there are several time series, OpenBDLM needs to know if there are dependencies between the time series.
%Typing \colorbox{light-gray}{\lstinline[basicstyle = \mlttfamily \small, backgroundcolor = \color{light-gray}]!2!} for the first time series, and \colorbox{light-gray}{\lstinline[basicstyle = \mlttfamily \small, backgroundcolor = \color{light-gray}]!0!} for the second time series means that the model will consider the irreversible components (if any) of the second (temperature) time series as covariates to describe the irreversible patterns observed in the first (displacement) time series.
%Then, OpenBDLM asks for the type of block component for each time series. 
%Type \colorbox{light-gray}{\lstinline[basicstyle = \mlttfamily \small, backgroundcolor = \color{light-gray}]![11 41]!} for the displacement time series and \colorbox{light-gray}{\lstinline[basicstyle = \mlttfamily \small, backgroundcolor = \color{light-gray}]![11 31 31 41]!} for the temperature time series.
%The yearly and daily periodic patterns observed in the displacement time series are modelled using the dependence on the periodic components defined for modelling the temperature time series data.
%Note that because an autoregressive component is chosen for the displacement time series data, the time-dependent model error on the displacement is modelled using a dependence on the autoregressive component of the temperature time series data, as well as an independent autoregressive component.
%The  output on \MATLAB{} command window during interactive model configuration is presented in Listing~\ref{LST:OpenBDLMModelConfigureExample2}.
%Type $\dlsh$ to valid.
%The model is then built, a \lstinline[basicstyle = \mlttfamily \small, backgroundcolor = \color{light-gray}]!DATA_DISPTEMP.mat! binary data file, a \lstinline[basicstyle = \mlttfamily \small, backgroundcolor = \color{light-gray}]!CFG_DISPTEMP.m! configuration file, as well as a \lstinline[basicstyle = \mlttfamily \small, backgroundcolor = \color{light-gray}]!PROJ_DISPTEMP.mat! project file are created.
%The OpenBLDM main menu must appear on the \MATLAB{} command window (see Listing~\ref{LST:OpenBDLMMainMenu}).
%Type \colorbox{light-gray}{\lstinline[basicstyle = \mlttfamily \small, backgroundcolor = \color{light-gray}]!Q!} to save and quit.



% \begin{lstlisting}[ frame = single, basicstyle = \mlttfamily \small, caption = { \MATLAB{} command window output during model configuration}, label = LST:OpenBDLMModelConfigureExample2,  float =h!, linewidth=\linewidth, captionpos=b, breaklines=true]
%- Identifies dependence between time series; use [0] to indicate no dependence
%    time serie #1 depends on time series # >> [2]
%
%- Identifies dependence between time series; use [0] to indicate no dependence
%    time serie #2 depends on time series # >> [0]
%
%- How many model classes do you want for each time series? 
%     choice >> 1
%     
%     ------------------------------------
%          BDLM Component reference numbers
%     ------------------------------------
%     11: Local level 
%     12: Local trend 
%     13: Local acceleration 
%     21: Local level compatible with local trend 
%     22: Local level compatible with local acceleration 
%     23: Local trend compatible with local acceleration 
%     31: Periodic 
%     41: Autoregressive process (AR(1)) 
%     51: Kernel regression 
%     61: Level Intervention 
%     --------------------------------------
%
%- Identify components for time series #1; e.g. [11 31 41]
%     choice >> [11 41]
%
%- Identify components for time series #2; e.g. [11 31 41]
%     choice >> [11 31 31 41]
%
%     Building model...
%     Saving project...
%     Project saved in saved_projects/PROJ_Example_DISPTEMP.mat. 
%     Printing configuration file...
%     Saving data...
%     Database saved in data/mat/DATA_Example_DISPTEMP.mat 
%     Configuration file saved in config_files/CFG_Example_DISPTEMP.m. 
%\end{lstlisting}


%\subsection{Step 5: open the configuration file}
%
%After the data loading and the model configuration, a configuration file named \lstinline[basicstyle = \mlttfamily \small, backgroundcolor = \color{light-gray}]!CFG_Example_DISPTEMP.m! configuration file is automatically created and saved in ``config\_files'' folder.
%Open the configuration file from \MATLAB{} command line by typing  \colorbox{light-gray}{\lstinline[basicstyle = \mlttfamily \small, backgroundcolor = \color{light-gray}]!edit CFG_Example_DISPTEMP.m!}.
%%The first part of this configuration file as it should appear on the \MATLAB{} editor is shown in Listing~\ref{LST:CFGFileExample1}.
%The Model parameters section of the configuration file shows that the model totalizes 15 model parameters, that is 
%\begin{gather*}
%\bm\theta=\{\sigma_{w, \mathtt{D}}^{LL},  \phi^{AR}_{\mathtt{D}}, \sigma_{w,\mathtt{D}}^{AR} ,\sigma_{v,\mathtt{D}},  \\
% \sigma_{w, \mathtt{T}}^{LL}, p^{\text{PD1}}_{\mathtt{T}}, \sigma_{w,\mathtt{T}}^{\text{PD1}} , p^{\text{PD2}}_{T}, \sigma_{w, \mathtt{T}}^{\text{PD2}}, \phi^{AR}_{T}, \sigma_{w, \mathtt{T}}^{AR}, \sigma_{v,\mathtt{T}}, \phi^{\mathtt{D}|\mathtt{T}}_{PD1}, \phi^{\mathtt{D}|\mathtt{T}}_{PD2},  \phi^{\mathtt{D}|\mathtt{T}}_{AR}\}.
%\end{gather*}
%%The default value of the model parameters are assigned  using heuristic knowledge or computed from the data using statistics on the data.
%The default model parameters values are 
%\begin{gather*}
%\bm\theta^{\text{default}}=\{0, 0.75, 0.017, 0.0087, \\
%0, 365.24, 0, 1, 0, 0.75, 1.2905, 0.64526, 0.5, 0.5, 0.5 \}.
%\end{gather*}
%%In the same manner, default value for the initial hidden states are assigned using heuristic knowledge or computed using statistics on the data.
%The default initial hidden states mean  and covariance values are 
%\begin{align*}
% \bm \mu^{\text{default}}_{0} & = [	25.7  ,	0  ,   	16.7  	, 5     ,	0   ,  	5   ,  	0    , 	0        ]^{\intercal}, \text{and} \\
% \text{diag}(\bm\Sigma^{\text{default}}_{0})  & = [	0.122, 	0.0305,	666,   	666,   	666,   	666,   	666 ,  	167     ],
%\end{align*}
%respectively.
%%$\bm \mu^{\text{default}}_{0} = [	25.7  ,	0  ,   	16.7  	, 5     ,	0   ,  	5   ,  	0    , 	0        ]$, and $\text{diag}(\bm\Sigma^{\text{default}}_{0}) = [	0.122, 	0.0305,	666,   	666,   	666,   	666,   	666 ,  	167     ]$, respectively.
%In the Options section, change \lstinline[basicstyle = \mlttfamily \small, backgroundcolor = \color{light-gray}]!misc.options.MethodStateEstimation='kalman'! to \lstinline[basicstyle = \mlttfamily \small, backgroundcolor = \color{light-gray}]!misc.options.MethodStateEstimation='UD'!. 
%In this specific example, the presence of missing data requires the use of UD computations instead of the standard, default, Kalman computation.
%Note that the choice about UD or Kalman is problem dependent. 
%
%\subsection{Step 6: estimate the hidden states}
%
%Type \colorbox{light-gray}{\lstinline[basicstyle = \mlttfamily \small, backgroundcolor = \color{light-gray}]!OpenBDLM_main('CFG_Example_DISPTEMP.m');!} in the \MATLAB{} command line.
%Once, the main menu appears, type  \colorbox{light-gray}{\lstinline[basicstyle = \mlttfamily \small, backgroundcolor = \color{light-gray}]!3!}, then \colorbox{light-gray}{\lstinline[basicstyle = \mlttfamily \small, backgroundcolor = \color{light-gray}]!1!} to estimate the filtered hidden states using the default model parameters and default initial hidden states values.
%The value of the log-likelihood is $-3800091$.
%The estimated hidden states are presented in Figure~\ref{fig:DISPTEMPSIMDefaultDefaultExample2}.


%\subsection{Step 7: estimate the model parameters from the data}
%
%Type \colorbox{light-gray}{\lstinline[basicstyle = \mlttfamily \small, backgroundcolor = \color{light-gray}]!OpenBDLM_main('CFG_Example_DISPTEMP.m');!} in the \MATLAB{} command line.
%Once, the main menu appears, type  \colorbox{light-gray}{\lstinline[basicstyle = \mlttfamily \small, backgroundcolor = \color{light-gray}]!1!}, then \colorbox{light-gray}{\lstinline[basicstyle = \mlttfamily \small, backgroundcolor = \color{light-gray}]!1!} to estimate the model parameters using Newton-Raphson (type  \colorbox{light-gray}{\lstinline[basicstyle = \mlttfamily \small, backgroundcolor = \color{light-gray}]!1!} to use the Stochastic Gradient instead).
%The model parameters learning procedure should start (see for example Listing~\ref{LST:OpenBLDMModelParameterLearning}).
%Note that, by default, OpenBDLM considers that the parameters $\sigma_{w, \mathtt{D}}^{LL}$, $\sigma_{w, \mathtt{T}}^{LL}$, $p^{\text{PD1}}_{\mathtt{T}}$, $\sigma_{w, \mathtt{T}}^{\text{PD1}}$ , $p^{\text{PD2}}_{\mathtt{T}}$, $\sigma_{w,\mathtt{T}}^{\text{PD2}}$ are known.
%Therefore, there are nine model parameters to be learned from the data in this example.
%The estimation of the model parameters may take several hours.
%Therefore, press combinations \colorbox{light-gray}{\lstinline[basicstyle = \mlttfamily \small, backgroundcolor = \color{light-gray}]!Ctrl!} + \colorbox{light-gray}{\lstinline[basicstyle = \mlttfamily \small, backgroundcolor = \color{light-gray}]!c!} to abort the process.
%Once the algorithm is converged, the optimized model parameters values should be close to  \footnote{Note that it is possible to get slightly different value of parameters with the same performance.}
%\begin{gather*}
% \bm\theta^{\text{*}}=\{0, 0.97, 0.019, 7.42\times10^{-7}, 0, 365.2422, 0, 1, 0, 0.99, 0.43, 2.67\times10^{-5},  \\
% -0.011, 0.0711, 0.000292 \}
%\end{gather*}


%\subsection{Step 8: estimate the hidden states using the optimized model parameters. values}
%
%In the ``examples/Example\_DISPTEMP'' folder, there is a configuration file named \lstinline[basicstyle = \mlttfamily \small, backgroundcolor = \color{light-gray}]!CFG_Example_DISPTEMP_optim.m! that contains optimized model parameters estimated using the Newton-Raphson algorithm.
%Copy and paste \lstinline[basicstyle = \mlttfamily \small, backgroundcolor = \color{light-gray}]!CFG_Example_DISPTEMP_optim.m! from  the ``examples/Example\_DISPTEMP'' subfolder  to the ``config\_files'' folder.
%Type \colorbox{light-gray}{\lstinline[basicstyle = \mlttfamily \small, backgroundcolor = \color{light-gray}]!OpenBDLM_main('CFG_Example_DISPTEMP_optim.m');!} in the \MATLAB{} command line to load the configuration file  \lstinline[basicstyle = \mlttfamily \small, backgroundcolor = \color{light-gray}]!CFG_Example_DISPTEMP_optim.m!.
%Once the main menu appears, type  \colorbox{light-gray}{\lstinline[basicstyle = \mlttfamily \small, backgroundcolor = \color{light-gray}]!3!}, then \colorbox{light-gray}{\lstinline[basicstyle = \mlttfamily \small, backgroundcolor = \color{light-gray}]!1!} to estimate the filtered hidden states using the optimized model parameters and default initial hidden states values.
%The value of the log-likelihood is now $38085$.
%The estimated hidden states are presented in Figure~\ref{fig:DISPTEMPSIMOptimizedDefaultExample2}.

%\subsection{Step 9: estimate the initial hidden states}
%
%Type \colorbox{light-gray}{\lstinline[basicstyle = \mlttfamily \small, backgroundcolor = \color{light-gray}]!OpenBDLM_main('CFG_Example_DISPTEMP_optim.m');!} in the \MATLAB{} command line.
%Then, type  \colorbox{light-gray}{\lstinline[basicstyle = \mlttfamily \small, backgroundcolor = \color{light-gray}]!2!}, to optimize the initial hidden states value.
%The estimated initial hidden states mean and covariance values are 
%\begin{align*}
%\bm \mu^{*}_{0} & = [	 25.9  ,	-0.0595	, 5.45  	, 17.6  ,	-0.934	, 0.678 ,	0.41  ,	2.1]^{\intercal}, \text{and} \\
% \text{diag}(\bm\Sigma^{*}_{0}) & = [	3.71\times10^{-5},	0.000457	, 0.287 	, 0.263 ,	0.265 	,8.14\times10^{-5}	, \\
% & 8.14\times10^{-5}	, 0.71    ], 
% \end{align*}
% respectively.
%Once it is done, type  \colorbox{light-gray}{\lstinline[basicstyle = \mlttfamily \small, backgroundcolor = \color{light-gray}]!3!}, and then  \colorbox{light-gray}{\lstinline[basicstyle = \mlttfamily \small, backgroundcolor = \color{light-gray}]!1!} to compute the filtered hidden states using the optimized model parameters and optimized initial hidden states.
%The value of the log-likelihood is $38120$.
%The estimated hidden states are presented in Figure~\ref{fig:DISPTEMPSIMOptimizedOptimizedExample2}.


%\begin{figure*}[h!]
%\centering
%\begin{subfigure}{\linewidth}
%\includegraphics[width=0.9\linewidth]{./docfigs/Example_DISPTEMPSIM/default/DISP_ObservedPredicted.pdf}
%\caption{Observed and estimated displacement data} 
%\end{subfigure}
%\begin{subfigure}{\linewidth}
%\includegraphics[width=0.9\linewidth]{./docfigs/Example_DISPTEMPSIM/default/DISP_LL_1.pdf}
%\caption{Estimated displacement local level component.}
%\end{subfigure}
%\begin{subfigure}{\linewidth}
%\includegraphics[width=0.9\linewidth]{./docfigs/Example_DISPTEMPSIM/default/DISP_AR_2.pdf}
%\caption{Estimated displacement autoregressive component.}
%\end{subfigure}
%\end{figure*}
%\begin{figure*}[h!]
%\ContinuedFloat
%\begin{subfigure}{\linewidth}
%\includegraphics[width=0.9\linewidth]{./docfigs/Example_DISPTEMPSIM/default/TEMP_ObservedPredicted.pdf} 
%\caption{Observed and estimated temperature data}
%\end{subfigure}
%\begin{subfigure}{\linewidth}
%\includegraphics[width=0.9\linewidth]{./docfigs/Example_DISPTEMPSIM/default/TEMP_LL_1.pdf} 
%\caption{Estimated temperature local level component.}
%\end{subfigure}
%\begin{subfigure}{\linewidth}
%\includegraphics[width=0.9\linewidth]{./docfigs/Example_DISPTEMPSIM/default/TEMP_S1_2.pdf} 
%\caption{Estimated temperature yearly component (first hidden state)}
%\end{subfigure}
%\begin{subfigure}{\linewidth}
%\includegraphics[width=0.9\linewidth]{./docfigs/Example_DISPTEMPSIM/default/TEMP_S1_4.pdf} 
%\caption{Estimated temperature daily component (first hidden state)}
%\end{subfigure}
%\begin{subfigure}{\linewidth}
%\includegraphics[width=0.9\linewidth]{./docfigs/Example_DISPTEMPSIM/default/TEMP_AR_6.pdf} 
%\caption{Estimated temperature autoregressive component}
%\end{subfigure}
%\caption{Estimated results using OpenBDLM default model parameters and default initial hidden states. The hidden states are estimated from the data presented in Figure~\ref{fig:DataSummaryDefaultPreProcessed2}a. The solid line and shaded area represent the mean and standard deviation of the estimated hidden states, respectively.}
%\label{fig:DISPTEMPSIMDefaultDefaultExample2}
%\end{figure*}

%\begin{figure*}[h!]
%\centering
%\begin{subfigure}{\linewidth}
%\includegraphics[width=0.9\linewidth]{./docfigs/Example_DISPTEMPSIM/optim_param_default_initialhiddenstate/DISP_ObservedPredicted.pdf}
%\caption{Observed and estimated displacement data} 
%\end{subfigure}
%\begin{subfigure}{\linewidth}
%\includegraphics[width=0.9\linewidth]{./docfigs/Example_DISPTEMPSIM/optim_param_default_initialhiddenstate/DISP_LL_1.pdf}
%\caption{Estimated displacement local level component.}
%\end{subfigure}
%\begin{subfigure}{\linewidth}
%\includegraphics[width=0.9\linewidth]{./docfigs/Example_DISPTEMPSIM/optim_param_default_initialhiddenstate/DISP_AR_2.pdf}
%\caption{Estimated displacement autoregressive component.}
%\end{subfigure}
%\end{figure*}
%\begin{figure*}[h!]
%\ContinuedFloat
%\begin{subfigure}{\linewidth}
%\includegraphics[width=0.9\linewidth]{./docfigs/Example_DISPTEMPSIM/optim_param_default_initialhiddenstate/TEMP_ObservedPredicted.pdf} 
%\caption{Observed and estimated temperature data}
%\end{subfigure}
%\begin{subfigure}{\linewidth}
%\includegraphics[width=0.9\linewidth]{./docfigs/Example_DISPTEMPSIM/optim_param_default_initialhiddenstate/TEMP_LL_1.pdf} 
%\caption{Estimated temperature local level component.}
%\end{subfigure}
%\begin{subfigure}{\linewidth}
%\includegraphics[width=0.9\linewidth]{./docfigs/Example_DISPTEMPSIM/optim_param_default_initialhiddenstate/TEMP_S1_2.pdf} 
%\caption{Estimated temperature yearly component (first hidden state)}
%\end{subfigure}
%\begin{subfigure}{\linewidth}
%\includegraphics[width=0.9\linewidth]{./docfigs/Example_DISPTEMPSIM/optim_param_default_initialhiddenstate/TEMP_S1_4.pdf} 
%\caption{Estimated temperature daily component (first hidden state)}
%\end{subfigure}
%\begin{subfigure}{\linewidth}
%\includegraphics[width=0.9\linewidth]{./docfigs/Example_DISPTEMPSIM/optim_param_default_initialhiddenstate/TEMP_AR_6.pdf} 
%\caption{Estimated temperature autoregressive component}
%\end{subfigure}
%\caption{Estimated results using OpenBDLM optimized model parameters and default initial hidden states. The hidden states are estimated from the data presented in Figure~\ref{fig:DataSummaryDefaultPreProcessed2}a. The solid line and shaded area represent the mean and standard deviation of the estimated hidden states, respectively.}
%\label{fig:DISPTEMPSIMOptimizedDefaultExample2}
%\end{figure*}

\begin{figure*}[h!]
\centering
\begin{subfigure}{\linewidth}\centering
\includegraphics[width=0.9\linewidth]{./docfigs/Example_DISPTEMPSIM/optim_param_optim_initialhiddenstate/DISP_ObservedPredicted.pdf}
\caption{Observed and estimated displacement data} 
\end{subfigure}
\begin{subfigure}{\linewidth}\centering
\includegraphics[width=0.9\linewidth]{./docfigs/Example_DISPTEMPSIM/optim_param_optim_initialhiddenstate/DISP_LL_1.pdf}
\caption{Estimated displacement local level component.}
\end{subfigure}
\begin{subfigure}{\linewidth}\centering
\includegraphics[width=0.9\linewidth]{./docfigs/Example_DISPTEMPSIM/optim_param_optim_initialhiddenstate/DISP_AR_2.pdf}
\caption{Estimated displacement autoregressive component.}
\end{subfigure}
\end{figure*}
\begin{figure*}[h!]
\ContinuedFloat
\begin{subfigure}{\linewidth}\centering
\includegraphics[width=0.9\linewidth]{./docfigs/Example_DISPTEMPSIM/optim_param_optim_initialhiddenstate/TEMP_ObservedPredicted.pdf} 
\caption{Observed and estimated temperature data}
\end{subfigure}
\begin{subfigure}{\linewidth}\centering
\includegraphics[width=0.9\linewidth]{./docfigs/Example_DISPTEMPSIM/optim_param_optim_initialhiddenstate/TEMP_LL_1.pdf} 
\caption{Estimated temperature local level component.}
\end{subfigure}
\begin{subfigure}{\linewidth}\centering
\includegraphics[width=0.9\linewidth]{./docfigs/Example_DISPTEMPSIM/optim_param_optim_initialhiddenstate/TEMP_S1_2.pdf} 
\caption{Estimated temperature yearly component (first hidden state)}
\end{subfigure}
\begin{subfigure}{\linewidth}\centering
\includegraphics[width=0.9\linewidth]{./docfigs/Example_DISPTEMPSIM/optim_param_optim_initialhiddenstate/TEMP_S1_4.pdf} 
\caption{Estimated temperature daily component (first hidden state)}
\end{subfigure}
\begin{subfigure}{\linewidth}\centering
\includegraphics[width=0.9\linewidth]{./docfigs/Example_DISPTEMPSIM/optim_param_optim_initialhiddenstate/TEMP_AR_6.pdf} 
\caption{Estimated temperature autoregressive component}
\end{subfigure}
\caption{Estimated results using OpenBDLM with the optimized model parameters and initial hidden states. The hidden states are estimated from the data presented in Figure~\ref{fig:DataSummaryDefaultPreProcessed2}a. The solid line and shaded area represent the mean and standard deviation of the estimated hidden states, respectively.}
\label{fig:DISPTEMPSIMOptimizedOptimizedExample2}
\end{figure*}