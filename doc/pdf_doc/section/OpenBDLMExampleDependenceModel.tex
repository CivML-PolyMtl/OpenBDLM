\section{Step-by-step example: dependence model between two time-series}
\label{S:ExampleDispTemp}


This example uses two time series data that mimics displacement and temperature data measured on a bridge (synthetic data). 

\subsection{Step 1: start a project}

First, choose the interactive tool by typing \colorbox{light-gray}{\lstinline[basicstyle = \mlttfamily \small, backgroundcolor = \color{light-gray}]!0!}.
Secondly, provide a project name (e.i \lstinline[basicstyle = \mlttfamily \small, backgroundcolor = \color{light-gray}]!Example_DISPTEMP!).
Then, answer \colorbox{light-gray}{\lstinline[basicstyle = \mlttfamily \small, backgroundcolor = \color{light-gray}]!no!} to indicate that you are not concerned with creating synthetic data.
The next step is to type \colorbox{light-gray}{\lstinline[basicstyle = \mlttfamily \small, backgroundcolor = \color{light-gray}]!0!} to indicate that you aim to load new data.

\subsection{Step 2: load the data}

\begin{figure*}[h!]
\centering
\begin{subfigure}{\linewidth}
\includegraphics[width=0.9\linewidth]{./docfigs/Example_DISPTEMPSIM/raw/ALL_AMPLITUDES.pdf} 
\caption{Amplitude}
\end{subfigure}
\begin{subfigure}{\linewidth}
\includegraphics[width=0.9\linewidth]{./docfigs/Example_DISPTEMPSIM/raw/ALL_TIMESTEPS.pdf}
\caption{Timestep}
\end{subfigure}
\begin{subfigure}{\linewidth}
\includegraphics[width=0.9\linewidth]{./docfigs/Example_DISPTEMPSIM/raw/AVAILABILITY.pdf}
\caption{Availability}
\end{subfigure}
\caption{Data used in Section~\ref{S:ExampleDispTemp}}.
\label{fig:DataSummaryRaw2}
\end{figure*}

At this stage, a graphical user interface should appear on screen. 
Browse the ``examples/Example\_DISPTEMP'' folder to select the csv files named   \lstinline[basicstyle = \mlttfamily \small, backgroundcolor = \color{light-gray}]!Example_DISPTEMP_DISP.csv! and  \lstinline[basicstyle = \mlttfamily \small, backgroundcolor = \color{light-gray}]!Example_DISPTEMP_TEMP.csv!.
Then, click on the Add button, and then the Done button, as highlighted in Figure~\ref{fig:DataLoadingUIPickFileExample1}.
You will notice that some basic information regarding the loaded time-series, such as the time series index, the reference name and the number of data points are now displayed in the \MATLAB{} command window, as depicted in Listing~\ref{LST:OpenBDLMDataAvailabilityProcessedExample2}.
At this time, three \MATLAB{} figures as those represented in Figure~\ref{fig:DataSummary1} should popup on screen.
The first figure represents in red the data amplitude of each time series data; the second figure represents the data timestep, and the last figure the data availability.
The figures show that data points exist between August 2013 and October 2015 (see Figure~\ref{fig:DataSummaryRaw2}a).
The timestep is non-uniform; it varies from 1 hour to 25 hours (see Figure~\ref{fig:DataSummaryRaw2}b). 
The timestep vector is not identical on each time series. 
It means that the time-series are not synchronized between each other.
The most frequent (i.e referent) time step is 1 hour for both time-series.
There is no missing data on the displacement time-series, but there are missing data on the temperature time series as indicated by the red crosses on the Figure~\ref{fig:DataSummaryRaw2}c.
Each red cross indicate the presence of Not a Number (NaN) value in the time series data.

 \begin{lstlisting}[ frame = single, basicstyle = \mlttfamily \small, caption = { \MATLAB{} command window output after selected data files.} from \MATLAB{} command line, label = LST:OpenBDLMDataAvailabilityRawExample2,  float =h!, linewidth=\linewidth, captionpos=b]
- Data available: 
 
     Time series number #      Reference name            Size                     	
     -------------------------------------------------------------------
     1                         DISP                      [19366x1]                	
     2                         TEMP                      [19042x1]                	
     -------------------------------------------------------------------
\end{lstlisting}


\subsection{Step 3: edit and pre-process the data}

The next step of the analysis consists in editing and preprocessing the data.
The data editing and preprocessing menu as depicted in Listing~\ref{LST:Editing_menu} should appear on the \MATLAB{} command window.
In this example, there are two time-series which are not synchronized between each other.
Therefore, a pre-processing is required.
Custom pre-processing can be done using the data editing and preprocessing OpenBDLM tools (see Listing~\ref{LST:Editing_menu}).
In this example, we are interested in using the default OpenBDLM preprocessing.
In such case, the option \colorbox{light-gray}{\lstinline[basicstyle = \mlttfamily \small, backgroundcolor = \color{light-gray}]!7!} will automatically synchronized the two time-series between each other and create a new dataset.
After typing  \colorbox{light-gray}{\lstinline[basicstyle = \mlttfamily \small, backgroundcolor = \color{light-gray}]!7!}, a figure should popup on the screen that shows the processed data which are now synchronized between each other (see Figure~\ref{fig:DataSummaryDefaultPreProcessed2}). 
The displacement and temperature time series data now share same timestep vector, and therefore they both have the same number of data samples, as shown in the Listing~\ref{LST:OpenBDLMDataAvailabilityProcessedExample2}.

 \begin{lstlisting}[ frame = single, basicstyle = \mlttfamily \small, caption = { \MATLAB{} command window output after selected data files}, label = LST:OpenBDLMDataAvailabilityProcessedExample2,  float =h!, linewidth=\linewidth, captionpos=b]
- Data available: 
 
     Time series number #      Reference name            Size                     	
     -------------------------------------------------------------------
     1                         DISP                      [19366x1]                	
     2                         TEMP                      [19366x1]                	
     -------------------------------------------------------------------
\end{lstlisting}


\begin{figure*}[h!]
\centering
\begin{subfigure}{\linewidth}
\includegraphics[width=0.9\linewidth]{./docfigs/Example_DISPTEMPSIM/preprocessed_default/ALL_AMPLITUDES.pdf} 
\caption{Amplitude}
\end{subfigure}
\begin{subfigure}{\linewidth}
\includegraphics[width=0.9\linewidth]{./docfigs/Example_DISPTEMPSIM/preprocessed_default/ALL_TIMESTEPS.pdf}
\caption{Timestep}
\end{subfigure}
\begin{subfigure}{\linewidth}
\includegraphics[width=0.9\linewidth]{./docfigs/Example_DISPTEMPSIM/preprocessed_default/AVAILABILITY.pdf}
\caption{Availability}
\end{subfigure}
\caption{Data used in Section~\ref{S:ExampleDispTemp}}.
\label{fig:DataSummaryDefaultPreProcessed2}
\end{figure*}



\subsection{Step 4: configure the model}

The next step is to configure the model.
First, the program requests the number of model class.
In this exemple, the time series data looks stationary and we are not interested in anomaly detection, and therefore we type \colorbox{light-gray}{\lstinline[basicstyle = \mlttfamily \small, backgroundcolor = \color{light-gray}]!1!}.
Secondly, because there are several time series, OpenBDLM needs to know if there are dependencies between the time-series.
Typing \colorbox{light-gray}{\lstinline[basicstyle = \mlttfamily \small, backgroundcolor = \color{light-gray}]!2!} for the first time-series, and \colorbox{light-gray}{\lstinline[basicstyle = \mlttfamily \small, backgroundcolor = \color{light-gray}]!0!} for the second time series means that the model will consider the irreversible components (if any) of the second (temperature) time-series as covariates to describe the irreversible patterns observed in the first (displacement) time-series.
Then, OpenBDLM asks for the type of block component for each time-series. 
Type \colorbox{light-gray}{\lstinline[basicstyle = \mlttfamily \small, backgroundcolor = \color{light-gray}]![11 41]!} for the displacement time series and \colorbox{light-gray}{\lstinline[basicstyle = \mlttfamily \small, backgroundcolor = \color{light-gray}]![11 31 31 41]!} for the temperature time series.
The yearly and daily periodic patterns observed in the displacement time-series are modelled using the dependence on the periodic components defined for modelling the temperature time series data.
Note that because an autoregressive component is chosen for the displacement time series data, the time-dependent model error on the displacement is modelled using a dependence on the autoregressive component of the temperature time series data, as well as an independent autoregressive component.
The  output on \MATLAB{} command window during interactive model configuration is presented in Listing~\ref{LST:OpenBDLMModelConfigureExample2}.
Type $\dlsh$ to valid.
The model is then built, a \lstinline[basicstyle = \mlttfamily \small, backgroundcolor = \color{light-gray}]!DATA_DISPTEMP.mat! binary data file, a \lstinline[basicstyle = \mlttfamily \small, backgroundcolor = \color{light-gray}]!CFG_DISPTEMP.m! configuration file, as well as a \lstinline[basicstyle = \mlttfamily \small, backgroundcolor = \color{light-gray}]!PROJ_DISPTEMP.mat! project file are created.
The OpenBLDM main menu must appear on the \MATLAB{} command window (see Listing~\ref{LST:OpenBDLMMainMenu}).
Type \colorbox{light-gray}{\lstinline[basicstyle = \mlttfamily \small, backgroundcolor = \color{light-gray}]!Q!} to save and quit.



 \begin{lstlisting}[ frame = single, basicstyle = \mlttfamily \small, caption = { \MATLAB{} command window output during model configuration}, label = LST:OpenBDLMModelConfigureExample2,  float =h!, linewidth=\linewidth, captionpos=b, breaklines=true]
- Identifies dependence between time series; use [0] to indicate no dependence
    time serie #1 depends on time series # >> [2]

- Identifies dependence between time series; use [0] to indicate no dependence
    time serie #2 depends on time series # >> [0]

- How many model classes do you want for each time-series? 
     choice >> 1
     
     ------------------------------------
          BDLM Component reference numbers
     ------------------------------------
     11: Local level 
     12: Local trend 
     13: Local acceleration 
     21: Local level compatible with local trend 
     22: Local level compatible with local acceleration 
     23: Local trend compatible with local acceleration 
     31: Periodic 
     41: Autoregressive process (AR(1)) 
     51: Kernel regression 
     61: Level Intervention 
     --------------------------------------

- Identify components for time series #1; e.g. [11 31 41]
     choice >> [11 41]

- Identify components for time series #2; e.g. [11 31 41]
     choice >> [11 31 31 41]

     Building model...
     Saving project...
     Project saved in saved_projects/PROJ_Example_DISPTEMP.mat. 
     Printing configuration file...
     Saving data...
     Database saved in data/mat/DATA_Example_DISPTEMP.mat 
     Configuration file saved in config_files/CFG_Example_DISPTEMP.m. 
\end{lstlisting}


\subsection{Step 5: open the configuration file}

After the data loading and the model configuration, a configuration file named \lstinline[basicstyle = \mlttfamily \small, backgroundcolor = \color{light-gray}]!CFG_Example_DISPTEMP.m! configuration file is automatically created and saved in ``config\_files'' folder.
Open the configuration file from \MATLAB{} command line by typing  \colorbox{light-gray}{\lstinline[basicstyle = \mlttfamily \small, backgroundcolor = \color{light-gray}]!edit CFG_Example_DISPTEMP.m!}.
%The first part of this configuration file as it should appear on the \MATLAB{} editor is shown in Listing~\ref{LST:CFGFileExample1}.
The Model parameters section of the configuration file shows that the model totalizes 15 model parameters, that is 
\begin{gather*}
\bm\theta=\{\sigma_{w, \text{D}}^{LL},  \phi^{AR}_{\text{D}}, \sigma_{w,\text{D}}^{AR} ,\sigma_{v,\text{D}},  \\
 \sigma_{w, \text{T}}^{LL}, p^{\text{PD1}}_{\text{T}}, \sigma_{w,\text{T}}^{\text{PD1}} , p^{\text{PD2}}_{T}, \sigma_{w, \text{T}}^{\text{PD2}}, \phi^{AR}_{T}, \sigma_{w, \text{T}}^{AR}, \sigma_{v,\text{T}}, \phi^{\text{D}|\text{T}}_{PD1}, \phi^{\text{D}|\text{T}}_{PD2},  \phi^{\text{D}|\text{T}}_{AR}\}.
\end{gather*}
%The default value of the model parameters are assigned  using heuristic knowledge or computed from the data using statistics on the data.
The default model parameters values are 
\begin{gather*}
\bm\theta^{\text{default}}=\{0, 0.75, 0.017, 0.0087, \\
0, 365.24, 0, 1, 0, 0.75, 1.2905, 0.64526, 0.5, 0.5, 0.5 \}.
\end{gather*}
%In the same manner, default value for the initial hidden states are assigned using heuristic knowledge or computed using statistics on the data.
The default initial hidden states mean  and covariance values are 
\begin{align*}
 \bm \mu^{\text{default}}_{0} & = [	25.7  ,	0  ,   	16.7  	, 5     ,	0   ,  	5   ,  	0    , 	0        ]^{\intercal}, \text{and} \\
 \text{diag}(\bm\Sigma^{\text{default}}_{0})  & = [	0.122, 	0.0305,	666,   	666,   	666,   	666,   	666 ,  	167     ],
\end{align*}
respectively.
%$\bm \mu^{\text{default}}_{0} = [	25.7  ,	0  ,   	16.7  	, 5     ,	0   ,  	5   ,  	0    , 	0        ]$, and $\text{diag}(\bm\Sigma^{\text{default}}_{0}) = [	0.122, 	0.0305,	666,   	666,   	666,   	666,   	666 ,  	167     ]$, respectively.
In the Options section, change \lstinline[basicstyle = \mlttfamily \small, backgroundcolor = \color{light-gray}]!misc.options.MethodStateEstimation='kalman'! to \lstinline[basicstyle = \mlttfamily \small, backgroundcolor = \color{light-gray}]!misc.options.MethodStateEstimation='UD'!. 
In this specific example, the presence of missing data requires the use of UD computations instead of the standard, default, Kalman computation.
Note that the choice about UD or Kalman is problem dependent. 

\subsection{Step 6: estimate the hidden states}

Type \colorbox{light-gray}{\lstinline[basicstyle = \mlttfamily \small, backgroundcolor = \color{light-gray}]!OpenBDLM_main('CFG_Example_DISPTEMP.m');!} in the \MATLAB{} command line.
Once, the main menu appears, type  \colorbox{light-gray}{\lstinline[basicstyle = \mlttfamily \small, backgroundcolor = \color{light-gray}]!3!}, then \colorbox{light-gray}{\lstinline[basicstyle = \mlttfamily \small, backgroundcolor = \color{light-gray}]!1!} to estimate the filtered hidden states using the default model parameters and default initial hidden states values.
The value of the log-likelihood is $-3800091$.
The estimated hidden states are presented in Figure~\ref{fig:DISPTEMPSIMDefaultDefaultExample2}.


\subsection{Step 7: estimate the model parameters from the data}

Type \colorbox{light-gray}{\lstinline[basicstyle = \mlttfamily \small, backgroundcolor = \color{light-gray}]!OpenBDLM_main('CFG_Example_DISPTEMP.m');!} in the \MATLAB{} command line.
Once, the main menu appears, type  \colorbox{light-gray}{\lstinline[basicstyle = \mlttfamily \small, backgroundcolor = \color{light-gray}]!1!}, then \colorbox{light-gray}{\lstinline[basicstyle = \mlttfamily \small, backgroundcolor = \color{light-gray}]!1!} to estimate the model parameters using Newton-Raphson (type  \colorbox{light-gray}{\lstinline[basicstyle = \mlttfamily \small, backgroundcolor = \color{light-gray}]!1!} to use the Stochastic Gradient instead).
The model parameters learning procedure should start (see for example Listing~\ref{LST:OpenBLDMModelParameterLearning}).
Note that, by default, OpenBDLM considers that the parameters $\sigma_{w, \text{D}}^{LL}$, $\sigma_{w, \text{T}}^{LL}$, $p^{\text{PD1}}_{\text{T}}$, $\sigma_{w, \text{T}}^{\text{PD1}}$ , $p^{\text{PD2}}_{\text{T}}$, $\sigma_{w,\text{T}}^{\text{PD2}}$ are known.
Therefore, there are nine model parameters to be learned from the data in this example.
The estimation of the model parameters may take several hours.
Therefore, press combinations \colorbox{light-gray}{\lstinline[basicstyle = \mlttfamily \small, backgroundcolor = \color{light-gray}]!Ctrl!} + \colorbox{light-gray}{\lstinline[basicstyle = \mlttfamily \small, backgroundcolor = \color{light-gray}]!c!} to abort the process.
Once the algorithm is converged, the optimized model parameters values should be close to  \footnote{Note that it is possible to get slightly different value of parameters with the same performance.}
\begin{gather*}
 \bm\theta^{\text{*}}=\{0, 0.97, 0.019, 7.42\times10^{-7}, 0, 365.2422, 0, 1, 0, 0.99, 0.43, 2.67\times10^{-5},  \\
 -0.011, 0.0711, 0.000292 \}
\end{gather*}


\subsection{Step 8: estimate the hidden states using the optimized model parameters. values}

In the ``examples/Example\_DISPTEMP'' folder, there is a configuration file named \lstinline[basicstyle = \mlttfamily \small, backgroundcolor = \color{light-gray}]!CFG_Example_DISPTEMP_optim.m! that contains optimized model parameters estimated using the Newton-Raphson algorithm.
Copy and paste \lstinline[basicstyle = \mlttfamily \small, backgroundcolor = \color{light-gray}]!CFG_Example_DISPTEMP_optim.m! from  the ``examples/Example\_DISPTEMP'' subfolder  to the ``config\_files'' folder.
Type \colorbox{light-gray}{\lstinline[basicstyle = \mlttfamily \small, backgroundcolor = \color{light-gray}]!OpenBDLM_main('CFG_Example_DISPTEMP_optim.m');!} in the \MATLAB{} command line to load the configuration file  \lstinline[basicstyle = \mlttfamily \small, backgroundcolor = \color{light-gray}]!CFG_Example_DISPTEMP_optim.m!.
Once the main menu appears, type  \colorbox{light-gray}{\lstinline[basicstyle = \mlttfamily \small, backgroundcolor = \color{light-gray}]!3!}, then \colorbox{light-gray}{\lstinline[basicstyle = \mlttfamily \small, backgroundcolor = \color{light-gray}]!1!} to estimate the filtered hidden states using the optimized model parameters and default initial hidden states values.
The value of the log-likelihood is now $38085$.
The estimated hidden states are presented in Figure~\ref{fig:DISPTEMPSIMOptimizedDefaultExample2}.

\subsection{Step 9: estimate the initial hidden states}

Type \colorbox{light-gray}{\lstinline[basicstyle = \mlttfamily \small, backgroundcolor = \color{light-gray}]!OpenBDLM_main('CFG_Example_DISPTEMP_optim.m');!} in the \MATLAB{} command line.
Then, type  \colorbox{light-gray}{\lstinline[basicstyle = \mlttfamily \small, backgroundcolor = \color{light-gray}]!2!}, to optimize the initial hidden states value.
The estimated initial hidden states mean and covariance values are 
\begin{align*}
\bm \mu^{*}_{0} & = [	 25.9  ,	-0.0595	, 5.45  	, 17.6  ,	-0.934	, 0.678 ,	0.41  ,	2.1]^{\intercal}, \text{and} \\
 \text{diag}(\bm\Sigma^{*}_{0}) & = [	3.71\times10^{-5},	0.000457	, 0.287 	, 0.263 ,	0.265 	,8.14\times10^{-5}	, \\
 & 8.14\times10^{-5}	, 0.71    ], 
 \end{align*}
 respectively.
Once it is done, type  \colorbox{light-gray}{\lstinline[basicstyle = \mlttfamily \small, backgroundcolor = \color{light-gray}]!3!}, and then  \colorbox{light-gray}{\lstinline[basicstyle = \mlttfamily \small, backgroundcolor = \color{light-gray}]!1!} to compute the filtered hidden states using the optimized model parameters and optimized initial hidden states.
The value of the log-likelihood is $38120$.
The estimated hidden states are presented in Figure~\ref{fig:DISPTEMPSIMOptimizedOptimizedExample2}.


\begin{figure*}[h!]
\centering
\begin{subfigure}{\linewidth}
\includegraphics[width=0.9\linewidth]{./docfigs/Example_DISPTEMPSIM/default/DISP_ObservedPredicted.pdf}
\caption{Observed and estimated displacement data} 
\end{subfigure}
\begin{subfigure}{\linewidth}
\includegraphics[width=0.9\linewidth]{./docfigs/Example_DISPTEMPSIM/default/DISP_LL_1.pdf}
\caption{Estimated displacement local level component.}
\end{subfigure}
\begin{subfigure}{\linewidth}
\includegraphics[width=0.9\linewidth]{./docfigs/Example_DISPTEMPSIM/default/DISP_AR_2.pdf}
\caption{Estimated displacement autoregressive component.}
\end{subfigure}
\end{figure*}
\begin{figure*}[h!]
\ContinuedFloat
\begin{subfigure}{\linewidth}
\includegraphics[width=0.9\linewidth]{./docfigs/Example_DISPTEMPSIM/default/TEMP_ObservedPredicted.pdf} 
\caption{Observed and estimated temperature data}
\end{subfigure}
\begin{subfigure}{\linewidth}
\includegraphics[width=0.9\linewidth]{./docfigs/Example_DISPTEMPSIM/default/TEMP_LL_1.pdf} 
\caption{Estimated temperature local level component.}
\end{subfigure}
\begin{subfigure}{\linewidth}
\includegraphics[width=0.9\linewidth]{./docfigs/Example_DISPTEMPSIM/default/TEMP_S1_2.pdf} 
\caption{Estimated temperature yearly component (first hidden state)}
\end{subfigure}
\begin{subfigure}{\linewidth}
\includegraphics[width=0.9\linewidth]{./docfigs/Example_DISPTEMPSIM/default/TEMP_S1_4.pdf} 
\caption{Estimated temperature daily component (first hidden state)}
\end{subfigure}
\begin{subfigure}{\linewidth}
\includegraphics[width=0.9\linewidth]{./docfigs/Example_DISPTEMPSIM/default/TEMP_AR_6.pdf} 
\caption{Estimated temperature autoregressive component}
\end{subfigure}
\caption{Estimated results using OpenBDLM default model parameters and default initial hidden states. The hidden states are estimated from the data presented in Figure~\ref{fig:DataSummaryDefaultPreProcessed2}a. The solid line and shaded area represent the mean and standard deviation of the estimated hidden states, respectively.}
\label{fig:DISPTEMPSIMDefaultDefaultExample2}
\end{figure*}

\begin{figure*}[h!]
\centering
\begin{subfigure}{\linewidth}
\includegraphics[width=0.9\linewidth]{./docfigs/Example_DISPTEMPSIM/optim_param_default_initialhiddenstate/DISP_ObservedPredicted.pdf}
\caption{Observed and estimated displacement data} 
\end{subfigure}
\begin{subfigure}{\linewidth}
\includegraphics[width=0.9\linewidth]{./docfigs/Example_DISPTEMPSIM/optim_param_default_initialhiddenstate/DISP_LL_1.pdf}
\caption{Estimated displacement local level component.}
\end{subfigure}
\begin{subfigure}{\linewidth}
\includegraphics[width=0.9\linewidth]{./docfigs/Example_DISPTEMPSIM/optim_param_default_initialhiddenstate/DISP_AR_2.pdf}
\caption{Estimated displacement autoregressive component.}
\end{subfigure}
\end{figure*}
\begin{figure*}[h!]
\ContinuedFloat
\begin{subfigure}{\linewidth}
\includegraphics[width=0.9\linewidth]{./docfigs/Example_DISPTEMPSIM/optim_param_default_initialhiddenstate/TEMP_ObservedPredicted.pdf} 
\caption{Observed and estimated temperature data}
\end{subfigure}
\begin{subfigure}{\linewidth}
\includegraphics[width=0.9\linewidth]{./docfigs/Example_DISPTEMPSIM/optim_param_default_initialhiddenstate/TEMP_LL_1.pdf} 
\caption{Estimated temperature local level component.}
\end{subfigure}
\begin{subfigure}{\linewidth}
\includegraphics[width=0.9\linewidth]{./docfigs/Example_DISPTEMPSIM/optim_param_default_initialhiddenstate/TEMP_S1_2.pdf} 
\caption{Estimated temperature yearly component (first hidden state)}
\end{subfigure}
\begin{subfigure}{\linewidth}
\includegraphics[width=0.9\linewidth]{./docfigs/Example_DISPTEMPSIM/optim_param_default_initialhiddenstate/TEMP_S1_4.pdf} 
\caption{Estimated temperature daily component (first hidden state)}
\end{subfigure}
\begin{subfigure}{\linewidth}
\includegraphics[width=0.9\linewidth]{./docfigs/Example_DISPTEMPSIM/optim_param_default_initialhiddenstate/TEMP_AR_6.pdf} 
\caption{Estimated temperature autoregressive component}
\end{subfigure}
\caption{Estimated results using OpenBDLM optimized model parameters and default initial hidden states. The hidden states are estimated from the data presented in Figure~\ref{fig:DataSummaryDefaultPreProcessed2}a. The solid line and shaded area represent the mean and standard deviation of the estimated hidden states, respectively.}
\label{fig:DISPTEMPSIMOptimizedDefaultExample2}
\end{figure*}

\begin{figure*}[h!]
\centering
\begin{subfigure}{\linewidth}
\includegraphics[width=0.9\linewidth]{./docfigs/Example_DISPTEMPSIM/optim_param_optim_initialhiddenstate/DISP_ObservedPredicted.pdf}
\caption{Observed and estimated displacement data} 
\end{subfigure}
\begin{subfigure}{\linewidth}
\includegraphics[width=0.9\linewidth]{./docfigs/Example_DISPTEMPSIM/optim_param_optim_initialhiddenstate/DISP_LL_1.pdf}
\caption{Estimated displacement local level component.}
\end{subfigure}
\begin{subfigure}{\linewidth}
\includegraphics[width=0.9\linewidth]{./docfigs/Example_DISPTEMPSIM/optim_param_optim_initialhiddenstate/DISP_AR_2.pdf}
\caption{Estimated displacement autoregressive component.}
\end{subfigure}
\end{figure*}
\begin{figure*}[h!]
\ContinuedFloat
\begin{subfigure}{\linewidth}
\includegraphics[width=0.9\linewidth]{./docfigs/Example_DISPTEMPSIM/optim_param_optim_initialhiddenstate/TEMP_ObservedPredicted.pdf} 
\caption{Observed and estimated temperature data}
\end{subfigure}
\begin{subfigure}{\linewidth}
\includegraphics[width=0.9\linewidth]{./docfigs/Example_DISPTEMPSIM/optim_param_optim_initialhiddenstate/TEMP_LL_1.pdf} 
\caption{Estimated temperature local level component.}
\end{subfigure}
\begin{subfigure}{\linewidth}
\includegraphics[width=0.9\linewidth]{./docfigs/Example_DISPTEMPSIM/optim_param_optim_initialhiddenstate/TEMP_S1_2.pdf} 
\caption{Estimated temperature yearly component (first hidden state)}
\end{subfigure}
\begin{subfigure}{\linewidth}
\includegraphics[width=0.9\linewidth]{./docfigs/Example_DISPTEMPSIM/optim_param_optim_initialhiddenstate/TEMP_S1_4.pdf} 
\caption{Estimated temperature daily component (first hidden state)}
\end{subfigure}
\begin{subfigure}{\linewidth}
\includegraphics[width=0.9\linewidth]{./docfigs/Example_DISPTEMPSIM/optim_param_optim_initialhiddenstate/TEMP_AR_6.pdf} 
\caption{Estimated temperature autoregressive component}
\end{subfigure}
\caption{Estimated results using OpenBDLM optimized model parameters and optimized initial hidden states. The hidden states are estimated from the data presented in Figure~\ref{fig:DataSummaryDefaultPreProcessed2}a. The solid line and shaded area represent the mean and standard deviation of the estimated hidden states, respectively.}
\label{fig:DISPTEMPSIMOptimizedOptimizedExample2}
\end{figure*}