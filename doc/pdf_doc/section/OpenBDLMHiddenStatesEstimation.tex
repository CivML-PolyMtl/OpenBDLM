\section{Hidden states estimation}
\label{S:HIDDENSTATESESTIMATION}

The estimation of the hidden states is the core of OpenBLDM.
%Each hidden state belongs to a block component used to build the model.
The hidden state estimation is performed recursively using the Kalman filter/smoother or the UD filter/smoother (see Section~\ref{SS:KFUD}).
Once a project is loaded, typing \colorbox{light-gray}{\lstinline[basicstyle = \mlttfamily \small ]!3!} opens the hidden state estimation menu (see Listing~\ref{LST:StateEstimationMenu}).

\begin{lstlisting}[ frame = single, basicstyle = \mlttfamily \small, caption = {OpenBDLM hidden states estimation menu}, label = LST:StateEstimationMenu ,  float =ht, linewidth=\linewidth, captionpos=b]
---------------------------------------------------
/    State estimation
---------------------------------------------------

     1 ->  Filter 
     2 ->  Smoother 

     Type R to return to the previous menu 

     choice >> 
\end{lstlisting}

Type  \colorbox{light-gray}{\lstinline[basicstyle = \mlttfamily \small ]!1!} runs the Kalman or UD filter, and typing \colorbox{light-gray}{\lstinline[basicstyle = \mlttfamily \small ]!2!} runs the Kalman or UD smoother.
The Kalman is the default state estimation method (see \lstinline[basicstyle = \mlttfamily \small ]!misc.options.MethodStateEstimation!), but UD computations are more stable in particular in the case of missing data or in case of a sudden increase of the time step.

\subsection{Hidden state estimation functions}

The hidden state estimation workflow is presented Figure~\ref{FIG:HiddenStateEstimationWorkflow}.
The OpenBLDM functions used for hidden state estimation are:

\begin{description}[style=unboxed]
\item[Pilote function for hidden state estimation] \leavevmode
  \begin{lstlisting}[ basicstyle = \mlttfamily \small, breaklines=true]
[data,model,estimation,misc]=piloteStateEstimation(data,model,estimation,misc)
  \end{lstlisting}

\item[Runs state estimation] \leavevmode
  \begin{lstlisting}[ basicstyle = \mlttfamily \small, breaklines=true]
[estimation]=StateEstimation(data,model,misc,varargin)
  \end{lstlisting}

\item[Runs switching Kalman filter for all time] \leavevmode
  \begin{lstlisting}[ basicstyle = \mlttfamily \small, breaklines=true]
[x,V,VV,S,loglik,U,D]=SwitchingKalmanFilter(data,model,misc)
  \end{lstlisting}

\item[Performs Rauch-Tung-Striebel switching smoother for all time] \leavevmode
  \begin{lstlisting}[ basicstyle = \mlttfamily \small, breaklines=true]
[x,V,VV,S,x_prior_smoothed,V_prior_smoothed,VV_prior_smoothed,S_prior_smoothed]=RTS_SwitchingKalmanSmoother(data,model,estimation)
  \end{lstlisting}

\item[Performs one step of the Kalman filter] \leavevmode
  \begin{lstlisting}[ basicstyle = \mlttfamily \small, breaklines=true]
[xnew,Vnew,VVnew,loglik]=KalmanFilter(A,C,Q,R,y,x,V,varargin)
  \end{lstlisting}

\item[Performs one step of the UD filter] \leavevmode
  \begin{lstlisting}[ basicstyle = \mlttfamily \small, breaklines=true]
[xnew,Vnew,VVnew,U_post,D_post,loglik]=UDFilter(A,C,Q,R,y,x,V,U_post,D_post)
  \end{lstlisting}

\item[Computes UD decomposition] \leavevmode
  \begin{lstlisting}[ basicstyle = \mlttfamily \small, breaklines=true]
[U,D] = myUD(mat,varargin)
  \end{lstlisting}
\end{description}



\begin{figure}[h]
  \centering
  \captionsetup{justification=centering}
\scalebox{0.8}{
\begin{tikzpicture}

\node[paraceladon](inputHSE){\begin{tabular}{c}  \lstinline[ basicstyle = \mlttfamily \small]!data! \\ \lstinline[ basicstyle = \mlttfamily \small]!model! \end{tabular}};
\node[esceladon](piloteSE)[below of = inputHSE, yshift = -1cm]{\phantom{} piloteStateEstimation.m \phantom{}};
\node[esceladon](SE)[below of = piloteSE, yshift = -1cm]{\phantom{} StateEstimation.m \phantom{}};
\node[testceladon](testSKSSKF)[below of = SE, yshift = -1.5cm]{\begin{tabular}{c} filter or  \\ smoother \\ ?  \end{tabular}};
\node[esceladon](RTS)[below of = testSKSSKF, yshift = -3.50cm, xshift = -2.5cm]{\begin{tabular}{c} RTS\_SwitchingKalman \\ Smoother.m \end{tabular}}; %{\phantom{} RTS\_SwitchingKalmanSmoother.m \phantom{}};
\node[esceladon](KFRTS)[left of = RTS, yshift = 0cm, xshift = -3.75cm]{\phantom{} KalmanFilter.m \phantom{}};
\node[esceladon](SKF)[below of = testSKSSKF , yshift = -3.50cm, xshift = 2.5cm]{\phantom{} SwitchingKalmanFilter.m \phantom{}};
\node[testceladon](testUDKF)[right of = SKF , yshift = 0cm, xshift = 3.5cm]{\begin{tabular}{c} KF or  \\ UD ? \end{tabular}};
\node[esceladon](UDF)[below of = testUDKF, yshift = -1.5cm]{\phantom{} UDFilter.m \phantom{}};
\node[esceladon](myUD)[right of = UDF, yshift = 0cm, xshift=1.5cm]{\phantom{} myUD.m \phantom{}};
\node[esceladon](KF)[below of = testUDKF, yshift = 3.5cm]{\phantom{} KalmanFilter.m \phantom{}};
\node[paraceladon](outputHSE)[below of = inputHSE, yshift = -13.5cm]{\lstinline[ basicstyle = \mlttfamily \small]!estimation!};
%
\path[->, draw, thick] (inputHSE)edge(piloteSE);
\path[->, draw, thick] (piloteSE)edge(SE);
\path[->, draw, thick] (SE)edge(testSKSSKF);
\path[->, draw, thick] (RTS)edge(KFRTS);
\path[->, draw, thick] (testSKSSKF.east) -| (2cm,-6.5cm) -| node[pos=0.25, above]{filter} (SKF);
\path[->, draw, thick] (testSKSSKF.west) -| (-2cm,-6.5cm) -| node[pos=0.25, above]{smoother} (RTS);
\path[->, draw, thick] (SKF.south) |- (0cm,-13cm) -|  (outputHSE.north);
\path[->, draw, thick] (RTS.south) |- (0cm,-13cm) -|  (outputHSE.north);
\path[->, draw, thick] (SKF)edge(testUDKF);
\path[->, draw, thick] (testUDKF)edge(KF);
\path[->, draw, thick] (testUDKF)edge node[anchor=center, above, rotate=90]{KF} (KF);
\path[->, draw, thick] (testUDKF)edge node[anchor=center, above, rotate=90]{UD} (UDF);
\path[->, draw, thick] (UDF)edge(myUD);
\end{tikzpicture} } 
\caption{Hidden states estimation workflow} \label{FIG:HiddenStateEstimationWorkflow}
\end{figure}


\subsection{Estimation of the initial hidden states}

OpenBDLM computes default values for the initial (at $t=0$) mean (\lstinline[ basicstyle = \mlttfamily \small]!model.initX!) and covariance (\lstinline[ basicstyle = \mlttfamily \small]!model.initV!) value, as well as the initial model probabilities (\lstinline[ basicstyle = \mlttfamily \small]!model.initS! . %only in the case of $\mathtt{S}=2$ model classes).
Those initial values are usually poor guesses, which may be refined using Kalman smoothing up to $t=0$.
The percent of data used to estimate initial hidden states using Kalman smoothing is controlled by the value provided in \lstinline[ basicstyle = \mlttfamily \small]!misc.options.DataPercent!.
From the OpenBDLM main menu, type \colorbox{light-gray}{\lstinline[basicstyle = \mlttfamily \small, backgroundcolor = \color{light-gray}]!2!}  in the \MATLAB{} command window to estimate the initial hidden states using Kalman smoothing.

The hidden state estimation workflow is presented Figure~\ref{FIG:InitialHiddenStateEstimationWorkflow}. 
The OpenBLDM functions used for initial hidden state estimation are:

\begin{description}[style=unboxed]
\item[Pilote function for initial state estimation] \leavevmode
  \begin{lstlisting}[ basicstyle = \mlttfamily \small, breaklines=true]
[data,model,estimation,misc]=piloteInitialStateEstimation(data,model,estimation,misc)
  \end{lstlisting}

\item[Runs state estimation] \leavevmode
  \begin{lstlisting}[ basicstyle = \mlttfamily \small, breaklines=true]
[estimation]=StateEstimation(data,model,misc,varargin)
  \end{lstlisting}

\item[Performs Rauch-Tung-Striebel switching smoother for all time] \leavevmode
  \begin{lstlisting}[ basicstyle = \mlttfamily \small, breaklines=true]
[x,V,VV,S,x_prior_smoothed,V_prior_smoothed,VV_prior_smoothed,S_prior_smoothed]=RTS_SwitchingKalmanSmoother(data,model,estimation)
  \end{lstlisting}

\item[Performs one step of the Kalman filter] \leavevmode
  \begin{lstlisting}[ basicstyle = \mlttfamily \small, breaklines=true]
[xnew,Vnew,VVnew,loglik]=KalmanFilter(A,C,Q,R,y,x,V,varargin)
  \end{lstlisting}

\end{description}




\begin{figure}[h]
  \centering
  \captionsetup{justification=centering}
\scalebox{0.8}{
\begin{tikzpicture}

\node[paraceladon](inputHSE){\begin{tabular}{c}  \lstinline[ basicstyle = \mlttfamily \small]!data! \\ \lstinline[ basicstyle = \mlttfamily \small]!model! \end{tabular}};
\node[esceladon](piloteSE)[below of = inputHSE, yshift = -1cm]{\phantom{} piloteInitialStateEstimation.m \phantom{}};
\node[esceladon](SE)[below of = piloteSE, yshift = -1cm]{\phantom{} StateEstimation.m \phantom{}};
\node[esceladon](RTS)[right of = SE, yshift = 0cm, xshift = 4.25cm]{\begin{tabular}{c} RTS\_SwitchingKalman \\ Smoother.m \end{tabular}};
\node[esceladon](KF)[right of = RTS, yshift = 0cm, xshift = 3.5cm]{\phantom{} KalmanFilter.m \phantom{}};
\node[paraceladon](outputHSE)[below of = inputHSE, yshift = -5cm]{\lstinline[ basicstyle = \mlttfamily \small]!model.initX,V,S!};
%
\path[->, draw, thick] (inputHSE)edge(piloteSE);
\path[->, draw, thick] (piloteSE)edge(SE);
\path[->, draw, thick] (SE)edge(RTS);
\path[->, draw, thick] (RTS)edge(KF);
\path[->, draw, thick] (SE)edge(outputHSE);
\end{tikzpicture} } 
\caption{Initial hidden states estimation workflow} \label{FIG:InitialHiddenStateEstimationWorkflow}
\end{figure}