\section{Generate synthetic data}
\label{S:SYNTHETIC}
The creation of synthetic data is possible using OpenBDLM.
The analysis of synthetic data is useful for validation, test and debugging purposes because the true value of the hidden states and model parameters are known.
OpenBDLM uses the transition model  of the state-space modelling approach (see Section~\ref{SS:LGSSM}) to create realistic synthetic data.
There are two ways for creating synthetic data using OpenBDLM:

\begin{itemize}
\item From the interactive tool
\item From an existing project.
\end{itemize}

\subsection{Generate synthetic data using the interactive tool}
The creation of the synthetic data from the interactive tool (option \colorbox{light-gray}{\lstinline[basicstyle = \mlttfamily \small, backgroundcolor = \color{light-gray}]!0!} from the starting menu) enables the creation of synthetic data from scratch. 
OpenBDLM requests the user to provide the number of time series, and to define the time vector (starting time, end time, timestep).
In the next step, the user has to define the time-series dependence (if applicable), to provide the number of model class, and to choose the block component for each time series, as well as model constrains between model classes (if applicable).
Default values for initial hidden states mean values and model parameters are automatically assigned for each block component.
In the case of two model classes, the synthetic baseline will switch between the first and the second model class according to the transition probability values (see Section~\ref{SS:THSKF}).
The amplitude of each synthetic anomaly (i.e. change of the local trend) is sampled randomly in a normal distribution of zero mean and standard deviation $\sigma_{w}^{12}$ as defined in the switching process noise transition matrix.
Alternately, the user may choose to create \emph{custom anomalies}.
In such a case, the beginning (in sample index), duration (in number of samples) and amplitude (in change of the local trend) of each anomaly is user specified.
The information about custom anomaly are stored in the field \lstinline[basicstyle = \mlttfamily \small, backgroundcolor = \color{light-gray}]!custom_anomalies! of the structure variable \lstinline[basicstyle = \mlttfamily \small, backgroundcolor = \color{light-gray}]!misc!:
\begin{itemize}
\item \lstinline[basicstyle = \mlttfamily \small, backgroundcolor = \color{light-gray}]!misc.custom_anomalies.start_custom_anomalies!: this field stores a $1\times \mathtt{A}$ vector of integers, where $\mathtt{A}$ is the total number of synthetic anomaly. Each value indicates the sample index of the anomaly start.
\item \lstinline[basicstyle = \mlttfamily \small, backgroundcolor = \color{light-gray}]!misc.custom_anomalies.duration_custom_anomalies!: this field stores a $1\times \mathtt{A}$ vector of integers, where $\mathtt{A}$ is the total number of synthetic anomaly. Each value indicates the anomaly duration in number of samples.
\item \lstinline[basicstyle = \mlttfamily \small, backgroundcolor = \color{light-gray}]!misc.custom_anomalies.amplitude_custom_anomalies!: this field stores a $1\times \mathtt{A}$ vector of real number, where $\mathtt{A}$ is the total number of synthetic anomaly. Each value indicates the amplitude of the anomaly in change of the local trend.
\end{itemize}
The synthetic data are saved in \lstinline[basicstyle = \mlttfamily \small, backgroundcolor = \color{light-gray}]!DATA_*.mat! and \lstinline[basicstyle = \mlttfamily \small, backgroundcolor = \color{light-gray}]!*.csv! data files, and a \lstinline[basicstyle = \mlttfamily \small, backgroundcolor = \color{light-gray}]!PROJ_*.mat! project file is created that stores the information about the model (structure variable \lstinline[basicstyle = \mlttfamily \small, backgroundcolor = \color{light-gray}]!model!), and the true hidden states (see structure variable \lstinline[basicstyle = \mlttfamily \small, backgroundcolor = \color{light-gray}]!estimation.ref!).





\begin{table}[h]
     \caption{Default value of model parameters and initial hidden states $\bm{\mu}_{0}$ and $\mathbf{\Sigma}_{0}$ for synthetic data generation.} 
     \centering
     \begin{tabular}{r|lp{3.1cm}p{4cm}}\thickhline
        & $\bm{\theta}$ & $\bm{\mu}_{0}$ & diag$(\mathbf{\Sigma}_{0})$ \\\cmidrule(lr){1-4}
    $\mathtt{LL}$   &  $\sigma_{w}^{\mathtt{LL}}=0$ &$[10]$ & $[0.1^{2}]$ \\
    $\mathtt{LT}$    & $\sigma_{w}^{\mathtt{LT}}=10^{-7}$ &  $[10, -0.1\times10^{-2}]$ & $[0.1^{2}, 0.1^{2}]$ \\
     $\mathtt{LA}$   & $\sigma_{w}^{\mathtt{LA}}=10^{-8}$  &  $[10, -0.1\times10^{-2} , -0.1\times10^{-5}]$ & $[0.1^{2}, 0.1^{2}, 0.1^{2}]$ \\
     $\mathtt{P}$  &  $p=[365.24, 1, 182.62] $, $\sigma_{w}^{\mathtt{P}}=0$  &$[10, 10]$ & $[0.2^{2}, 0.2^{2}]$  \\
     $\mathtt{KR}$  & $p=[365.24]$, $\ell=0.5$, $\sigma_{w,0}^{\mathtt{KR}}=\sigma_{w,1}^{\mathtt{KR}}=0$ &  $[$$-0.97$, $1.65$, $1.73$, $-1.91$, $0.23$, $0.37$, $-2.89$, $-0.22$, $0.73$, $-1.83$$]$ & $[$$0.01^{2}$, $0.01^{2}$, $0.01^{2}$, $0.01^{2}$, $0.01^{2}$, $0.01^{2}$, $0.01^{2}$, $0.01^{2}$, $0.01^{2}$, $0.01^{2}$$]$  \\  
         $\mathtt{AR}$   &  $\phi^{\mathtt{AR}}=0.75$, $\sigma_{w}^{\mathtt{AR}}=0.01$  &$[0]$ & $[0.1^{2}]$ \\\thickhline
     \end{tabular}
\label{table:defaultsynthetic}
\end{table}

\subsection{Generate synthetic data from an existing project}



Once a project is loaded, it is possible to create synthetic data from it (option \colorbox{light-gray}{\lstinline[basicstyle = \mlttfamily \small, backgroundcolor = \color{light-gray}]!16!} from the main menu (see Listing~\ref{LST:OpenBDLMMainMenu}).
The synthetic data time vector will be the same as the time vector in memory, and missing data will be replicated.
The model used to create the synthetic data will be the same as the model of the current project, including current initial hidden states as well as model parameters values.
The creation of synthetic data in this way is particularly useful to closely mimic real dataset.
The synthetic data are saved in \lstinline[basicstyle = \mlttfamily \small, backgroundcolor = \color{light-gray}]!DATA_new_*.mat! and \lstinline[basicstyle = \mlttfamily \small, backgroundcolor = \color{light-gray}]!*.csv! data files, and a \lstinline[basicstyle = \mlttfamily \small, backgroundcolor = \color{light-gray}]!PROJ_new_*.mat! new project file is created that stores the information about the model (structure variable \lstinline[basicstyle = \mlttfamily \small, backgroundcolor = \color{light-gray}]!model!), and the true hidden states (see structure variable \lstinline[basicstyle = \mlttfamily \small, backgroundcolor = \color{light-gray}]!estimation.ref!).






\subsection{Synthetic data generation functions}


The synthetic data creation workflow is presented Figure~\ref{FIG:SyntheticDataCreationWorkflow}. 
The OpenBLDM functions used for synthetic data creation are:

\begin{description}[style=unboxed]

\item[Pilot function for synthetic data creation] \leavevmode
  \begin{lstlisting}[ basicstyle = \mlttfamily \small, breaklines=true]
[data,model,estimation,misc]=piloteSimulateData(data,model,estimation,misc)
  \end{lstlisting}

\item[Creates synthetic data] \leavevmode
  \begin{lstlisting}[ basicstyle = \mlttfamily \small, breaklines=true]
[data,model,estimation,misc]=SimulateData(data,model,misc,varargin)
  \end{lstlisting}

\item[Create synthetic data from transition probabilities] \leavevmode
  \begin{lstlisting}[ basicstyle = \mlttfamily \small, breaklines=true]
[data, model, estimation, misc]=simulateDataFromTransitionProbabilities(data,model,misc)
  \end{lstlisting}

\item[Create synthetic data from custom anomalies (for two model classes only)] \leavevmode
  \begin{lstlisting}[ basicstyle = \mlttfamily \small, breaklines=true]
[data,model,estimation,misc]=simulateDataFromCustomAnomalies(data,model,misc)
  \end{lstlisting}

  \item[Models configuration for synthetic data (for synthetic data creation from interactive tool only)] \leavevmode
  \begin{lstlisting}[ basicstyle = \mlttfamily \small, breaklines=true]
[data,model,estimation,misc]=configureModelForDataSimulation(data,model,estimation,misc)
 \end{lstlisting}
 
 \item[Requests user inputs to define the number of synthetic time series to create (for synthetic data creation from interactive tool only)] \leavevmode
  \begin{lstlisting}[ basicstyle = \mlttfamily \small, breaklines=true]
  [data,misc]=defineDataLabels(data,misc)
 \end{lstlisting}
 
\item[Requests user inputs to define synthetic data time vector (for synthetic data creation from interactive tool only)] \leavevmode
  \begin{lstlisting}[ basicstyle = \mlttfamily \small, breaklines=true]
[data,misc]=defineTimestamps(data,misc)
 \end{lstlisting}

\end{description}



\begin{figure}[h]
  \centering
  \captionsetup{justification=centering}
\scalebox{0.8}{
\begin{tikzpicture}

\node[paralightgray](inputSDC){\begin{tabular}{c}  \lstinline[ basicstyle = \mlttfamily \small]!data! \\ \lstinline[ basicstyle = \mlttfamily \small]!model.param_properties! \\ \lstinline[ basicstyle = \mlttfamily \small]!model.initX, initV, initS! \end{tabular}};
\node[eslightgray](piloteSDC)[below of = inputSDC, yshift = -1cm]{\phantom{} piloteSimulateData.m \phantom{}};
\node[eslightgray](SDC)[below of = piloteSDC, yshift = -1cm]{\phantom{} SimulateData.m \phantom{}};
\node[testlightgray](testCustom)[below of = SDC, yshift = -1.5cm]{\begin{tabular}{c}  custom  \\ anomalies ?  \end{tabular}};
\node[eslightgray](SDCtransition)[below of = testCustom, yshift = -1.75cm, xshift = -3cm]{\begin{tabular}{c} SimulateData \\ FromTransitionProbabilities.m \end{tabular}};
\node[eslightgray](SDCcustom)[below of = testCustom , yshift = -1.75cm, xshift = 3cm]{\begin{tabular}{c} SimulateData \\ FromCustomAnomalies.m \end{tabular}};
\node[paralightgray](outputSDC)[below of = inputSDC, yshift = -10cm]{\lstinline[ basicstyle = \mlttfamily \small]!estimation.ref!};
%
\path[->, draw, thick] (inputSDC)edge(piloteSDC);
\path[->, draw, thick] (piloteSDC)edge(SDC);
\path[->, draw, thick] (SDC)edge(testCustom);
\path[->, draw, thick] (testCustom.east) -| (2cm,-6.5cm) -| node[pos=0.25, above]{yes} (SDCcustom);
\path[->, draw, thick] (testCustom.west) -| (-2cm,-6.5cm) -| node[pos=0.25, above]{no} (SDCtransition);
\path[->, draw, thick] (SDCtransition.south) |- (0cm,-10cm) -|  (outputSDC.north);
\path[->, draw, thick] (SDCcustom.south) |- (0cm,-10cm) -|  (outputSDC.north);
\end{tikzpicture} } 
\caption{Synthetic data creation workflow} \label{FIG:SyntheticDataCreationWorkflow}
\end{figure}