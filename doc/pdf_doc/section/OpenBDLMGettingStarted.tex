\section{Getting started}

\subsection{What is OpenBDLM ?}
\label{S:OPENBDLMWHATIS}

OpenBDLM is a \MATLAB{} open-source software developed to use Bayesian Dynamic Linear Models for long-term time series analysis (i.e time step in the order of one hour or higher).
OpenBDLM is capable of processing simultaneously several time series, enabling interpretation, monitoring and prediction over time.
OpenBDLM includes an anomaly detection tool which allows detecting abnormal behavior in a fully probabilistic framework.
In addition, OpenBDLM handles time series with missing data and non-uniform timestep vector.
OpenBDLM is available for download from GitHub at \url{https://github.com/CivML-PolyMtl/OpenBDLM}.\\

\noindent \textbf{Keywords}: time series analysis and forecasting, linear gaussian state-space models, time-series decomposition, anomaly detection, filtering, smoothing, bayesian analysis.

\subsection{Installing OpenBDLM}
\label{S:OPENBDLMINSTALLING}

The following instructions show how to download and setup OpenBDLM on your local machine for direct use, testing, and development purposes.
\subsubsection{Prerequisites}
\MATLAB{} (version 2016a or higher) installed on Mac OSX or Windows.\\

\noindent
The \MATLAB{}  \emph{Statistics and Machine Learning Toolbox} is required.

\subsubsection{Installation}
\begin{enumerate}
\item If an older OpenBDLM version is already installed, it is recommanded to remove from your \MATLAB{} path any previous OpenBDLM versions.
\item Download and extract the ZIP file from \url{https://github.com/CivML-PolyMtl/OpenBDLM} (or clone the git repository) in your working directory.
\item Add ``OpenBDLM-master'' folder and all its subfolders to your \MATLAB{}  path through one of the following two options:
\begin{itemize}
    \item using the ``Set Path'' dialog box in \MATLAB{}, or 
    \item by running  \lstinline[basicstyle = \mlttfamily \small, backgroundcolor = \color{light-gray}]!addpath! function from the \MATLAB{} command window
    \end{itemize}
\end{enumerate}


\subsection{Starting OpenBDLM}
\label{S:OPENBDLMGETTINGSTARTED}

\begin{itemize}
\item Set the current working directory to the folder ``OpenBDLM-master''
%\begin{center}
\item Type \colorbox{light-gray}{\lstinline[basicstyle = \mlttfamily \small, backgroundcolor = \color{light-gray}]!OpenBDLM_main;!}, and press the key enter $\dlsh$.
%\end{center}
in the \MATLAB{} command line.
\end{itemize}
The OpenBDLM starting menu should appear on the \MATLAB{} command window (see Listing~\ref{LST:OpenBDLMStartingMenu}).
Type \colorbox{light-gray}{\lstinline[basicstyle = \mlttfamily \small, backgroundcolor = \color{light-gray}]!Q!} in the command line, and press the key enter $\dlsh$ to quit the program.
%\begin{lstlisting}[ frame = single, basicstyle = \mlttfamily \small, backgroundcolor = \color{light-gray}, linewidth=\linewidth] %, caption = OpenBDLM call ]%, label = LST:OpenBDLMmain, captionpos=b]
%[data, model, estimation, misc] = OpenBDLM_main();
%\end{lstlisting}
\begin{lstlisting}[ frame = single, basicstyle = \mlttfamily \small, caption = {OpenBDLM starting menu on  \MATLAB{} command window when calling \lstinline!OpenBDLM\_main;!}, label = LST:OpenBDLMStartingMenu ,  float =ht, linewidth=\linewidth, captionpos=b]

------------------------------------------------
     Starting OpenBDLM_V1.0
------------------------------------------------
            Time series analysis using 
            Bayesian Dynamic Linear Models
------------------------------------------------
- Start a new project: 

     *      Enter a configuration filename 
     0   -> Interactive tool 

- Type D to Delete project(s), V for Version control, Q to Quit.

     choice >>
\end{lstlisting}


\subsection{Demo}

In the \MATLAB{} command line, type  
%\begin{lstlisting}[frame = single, basicstyle = \mlttfamily \small, backgroundcolor = \color{light-gray}, caption = Run the OpenBDLM demo, label = LST:OpenBDLMdemo, captionpos=b, linewidth=\linewidth]
%run_DEMO.m
%\end{lstlisting}
%\begin{center}
\colorbox{light-gray}{\lstinline[basicstyle = \mlttfamily \small, backgroundcolor = \color{light-gray}]!run_DEMO;! }
%\end{center}
 followed by pressing the key enter $\dlsh$ to run a demo. 
%\lstinline[basicstyle = \mlttfamily \small, backgroundcolor = \color{light-gray}]!run_DEMO.m! to run a little demo. 
Some messages on the \MATLAB{} command window show that the program has started (see Listing~\ref{LST:MessageDemo}).
Some figures as shown in Figure~\ref{fig:DataSummaryRaw4}, and Figure~\ref{fig:SYNTHETICOptimizedOptimizedExample4} should popup on the screen \footnote{The demo runs in batch the example described in Section~\ref{S:EXAMPLESYNTHETICDATA}.}


\begin{lstlisting}[frame = single, basicstyle = \mlttfamily \small, caption = {Output on \MATLAB{} command window when running \lstinline!run_DEMO.m!}, linewidth=\linewidth, label=LST:MessageDemo, ,captionpos=b, float = h] 
     Starting OpenBDLM_V1.0...
     Starting a new project...
     Building model...
     Simulating data...
     Plotting data...
     Saving database (binary format) ...
     Saving database (csv format) ...
     Saving project...
     Printing configuration file...
     Saving database (binary format) ...
     Saving project...
     Done ! See you soon !
\end{lstlisting}



\subsection{OpenBDLM main menu}
\label{S:OPENBDLMMAINMENU}

Once a project is loaded, the OpenBDLM main menu displays the list of actions which can be done (see Listing~\ref{LST:OpenBDLMMainMenu}). 
The OpenBDLM main menu appears each time a selected action is done; until the user types \colorbox{light-gray}{\lstinline[basicstyle = \mlttfamily \small ]!Q!} to save the project and quit the program.

%\begin{itemize}
%    \item option  \colorbox{light-gray}{\lstinline[basicstyle = \mlttfamily \small ]!1!}: estimates the model parameters
%    \item option  \colorbox{light-gray}{\lstinline[basicstyle = \mlttfamily \small ]!2!}: estimates the initial hidden states
%    \item option  \colorbox{light-gray}{\lstinline[basicstyle = \mlttfamily \small ]!3!}: estimate the hidden states
%    \item option  \colorbox{light-gray}{\lstinline[basicstyle = \mlttfamily \small ]!11!}: display the current model parameters values on the \MATLAB{} command line window and allows the user to modify the model parameters values
%     \item option  \colorbox{light-gray}{\lstinline[basicstyle = \mlttfamily \small ]!12!}: display the initial hidden states values on the \MATLAB{} command line window and allows the user to modify the initial hidden states values
%     \item option  \colorbox{light-gray}{\lstinline[basicstyle = \mlttfamily \small ]!13!}: display information about the current training period  on the \MATLAB{} command line window and allows the user to modify the training period
%     \item option  \colorbox{light-gray}{\lstinline[basicstyle = \mlttfamily \small ]!14!}: plot figures on the screen about the data in memory, the data availability, and the hidden states estimation results
%     \item option  \colorbox{light-gray}{\lstinline[basicstyle = \mlttfamily \small ]!15!}: display the content of the model matrices on the \MATLAB{} command line window at a specific time
%     \item option  \colorbox{light-gray}{\lstinline[basicstyle = \mlttfamily \small ]!16!}: create synthetic data from the current information in memory
%     \item option  \colorbox{light-gray}{\lstinline[basicstyle = \mlttfamily \small ]!17!}: export data and results as well as the figures in PNG, PDF and  \LaTeX{}   format
%     \item option  \colorbox{light-gray}{\lstinline[basicstyle = \mlttfamily \small ]!18!}: display the current options 
%    \end{itemize}

\begin{lstlisting}[ frame = single, basicstyle = \mlttfamily \small, caption = { OpenBLDM main menu}, label = LST:OpenBDLMMainMenu,  float =h!, linewidth=\linewidth, captionpos=b]
---------------------------------------------
/    OpenBDLM main menu. Choose from 
--------------------------------------------- 

     1  ->  Learn model parameters values 
     2  ->  Estimate initial hidden states values 
     3  ->  Estimate hidden states values 

     11 ->  Display and modify current model parameter values 
     12 ->  Display and modify current initial hidden states values 
     13 ->  Display and modify current training period 
     14 ->  Plots 
     15 ->  Display model matrices 
     16 ->  Create synthetic data 
     17 ->  Export
     18 ->  Display current options in configuration file format 

     Type Q to Save and Quit 

     choice >> 
\end{lstlisting}




