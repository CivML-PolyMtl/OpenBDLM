\section{Data editing and pre-processing}
\label{S:DATAEDITINGPREPROCESSING}

Data editing prepares the data for analysis.
In most cases, the set of available time-series is heterogeneous, in the sense that each time-series does not originate from the same system of acquisition.
Therefore, the raw data do not usually share the same time vectors.
This is an issue because BDLM techniques are not capable to analyze asynchronous time-series.
Therefore, when multiple time series are processed simultaneously, the main objective of data pre-processing is to synchronize the time-series. 
Data editing includes time series selection, data analysis time period selection, missing data removal, and data  resampling.
The time synchronization is performed automatically through the data editing process.

\begin{lstlisting}[ frame = single, basicstyle = \mlttfamily \small,  linewidth = \linewidth, caption = OpenBDLM data editing menu,  label = LST:Editing_menu, float = h!, ,captionpos=b]
- Data editing and preprocessing. Choose from:

     1  ->  Select time series
     2  ->  Select data analysis time period 
     3  ->  Remove missing data
     4  ->  Resample
     5  ->  Change synchronization options

     6  ->  Reset changes
     7  ->  Save changes and continue analysis

     choice >> 
\end{lstlisting}    

\subsection{Selection of time-series}
\label{SS:SelectionTimeSeries}

The selection of time-series allows selecting only a subset of the time series in memory.
The time series are automatically synchronized as time-series are added to (or removed from) the dataset.

\subsection{Selection of data analysis time period}
\label{SS:SelectionPeriodAnalysis}

The selection of the period of analysis allows selecting only the data between two dates, given as \textquotesingle YYYY-DD-MM\textquotesingle {} format.
If the second requested date exceeds the date corresponding to the last timestamp of the original dataset, padding with \lstinline[basicstyle = \mlttfamily \small ]!NaN! values are performed. 
The timestep for the \lstinline[basicstyle = \mlttfamily \small ]!NaN! padding must be provided by the user.

\subsection{Removing missing data}
\label{SS:MissingDataRemoval}

It is possible to control the maximum amount of \lstinline[basicstyle = \mlttfamily \small ]!NaN! missing data allowed at each time slice. 
The maximum amount of \lstinline[basicstyle = \mlttfamily \small ]!NaN! allowed at each time slice is given in percent with respect to the total number of time-series.
By default, the maximum amount of missing data is given by the value in \lstinline[basicstyle = \mlttfamily \small ]!misc.options.NaNThreshold!.

\subsection{Data resampling}
\label{SS:DataResampling}
Data resampling changes the sampling rate of the time-series according to a given timestep provided by the user. 
If the requested timestep is higher than the original data timestep, \lstinline[basicstyle = \mlttfamily \small ]!NaN! values are added.
Conversely, if the requested timestep is lower than the original  data timestep, OpenBDLM averages the data amplitude values within non-overlapping fixed time windows, each having the duration of the requested timestep.
The first time window starts at the first timestamp, and the new timestamps are assigned at the times corresponding to the middle of each time window.

\subsection{Time synchronization options}
\label{SS:synchronization}
By default, the time synchronization in OpenBDLM is done by adding \lstinline[basicstyle = \mlttfamily \small ]!NaN! values.
The time synchronization is controlled by the \lstinline[basicstyle = \mlttfamily \small ]!NaNThreshold! and \lstinline[basicstyle = \mlttfamily \small ]!tolerance! variables.
\lstinline[basicstyle = \mlttfamily \small ]!NaNThreshold!  is given in percent with respect to the total number of time-series.
The variable \lstinline[basicstyle = \mlttfamily \small ]!tolerance! gives the duration (in number of days) after which two timestamps are not considered equal.
The default values for \lstinline[basicstyle = \mlttfamily \small ]!NaNThreshold!  and \lstinline[basicstyle = \mlttfamily \small ]!tolerance! are given by \lstinline[basicstyle = \mlttfamily \small ]!misc.options.NaNThreshold! and \lstinline[basicstyle = \mlttfamily \small ]!misc.options.tolerance!. % $100$\% and $10^{-6}$ days, respectively. 

\subsection{Data editing functions}

The data editing workflow is presented Figure~\ref{FIG:DataEditingWorkflow}. The list of OpenBDLM functions used for data editing is:

\begin{description}[style=unboxed]
\item[Control script to edit dataset (selection, resampling, etc..)] \leavevmode
  \begin{lstlisting}[ basicstyle = \mlttfamily \small, breaklines=true]
[data,misc,dataFilename]=editData(data,misc,varargin)
 \end{lstlisting}

\item[Requests the user to select some time series] \leavevmode
  \begin{lstlisting}[ basicstyle = \mlttfamily \small, breaklines=true]
[data,misc]=chooseTimeSeries(data,misc)
 \end{lstlisting} 
 
\item[Creates a single time vector from a set of time series] \leavevmode
  \begin{lstlisting}[ basicstyle = \mlttfamily \small, breaklines=true]
[data,misc]=mergeTimeStampVectors(dataOrig,misc,varargin)
 \end{lstlisting} 
 
\item[Resamples dataset according to a given timestep] \leavevmode
  \begin{lstlisting}[ basicstyle = \mlttfamily \small, breaklines=true]
[data_resample,misc]=resampleData(data,misc,varargin)
 \end{lstlisting} 
 
 \item[Selects data between two dates] \leavevmode
  \begin{lstlisting}[ basicstyle = \mlttfamily \small, breaklines=true]
[data,misc]=selectTimePeriod(data,misc)
 \end{lstlisting} 
 
  \item[Computes timestep vector from timestamps vector] \leavevmode
  \begin{lstlisting}[ basicstyle = \mlttfamily \small, breaklines=true]
[timesteps]=computeTimeSteps(timestamps)
 \end{lstlisting} 
 
   \item[Display on screen the content of the data in memory] \leavevmode
  \begin{lstlisting}[ basicstyle = \mlttfamily \small, breaklines=true]
displayData(data,misc)
 \end{lstlisting} 
 
    \item[ Display the list of DATA\_*.mat files ] \leavevmode
  \begin{lstlisting}[ basicstyle = \mlttfamily \small, breaklines=true]
[FileInfo]= displayDataBinary(misc,varargin)
 \end{lstlisting} 
 
\end{description}

\begin{figure}[!h]
  \centering
  \captionsetup{justification=centering}
\scalebox{0.8}{
\begin{tikzpicture}

\node[paraamber](inputDataEditor){\lstinline[basicstyle = \mlttfamily \small ]!data!};
\node[esamber](editData)[below of = inputDataEditor,  yshift=-0.65cm, xshift = 0cm ]{ \phantom{} editData.m \phantom{} } ;
\node[esamber](editFunctions)[below of = editData, yshift = -1.25cm]{\begin{tabular}{c} chooseTimeSeries.m \\ selectTimePeriod.m \\ mergeTimeStampVectors.m \\ resampleData.m \end{tabular}};
\node[paraamber](DataMATFile)[below of =editFunctions, yshift=-1.75cm, xshift = 0cm]{\begin{tabular}{c} \lstinline[basicstyle = \mlttfamily \small ]!data.timestamps! \\ \lstinline[basicstyle = \mlttfamily \small ]!data.values! \\ \lstinline[basicstyle = \mlttfamily \small ]!data.labels! \\ \lstinline[basicstyle = \mlttfamily \small ]!DATA_*.MAT!  \end{tabular}};

\path[->, thick] (inputDataEditor)edge(editData);
\path[->, thick] (editData)edge(editFunctions);
\path[->, thick] (editFunctions)edge(DataMATFile);

\end{tikzpicture} } 
\caption{Data editing and pre-processing workflow} \label{FIG:DataEditingWorkflow}
\end{figure}

