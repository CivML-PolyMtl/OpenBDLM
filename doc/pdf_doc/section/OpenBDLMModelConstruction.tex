\subsection{Model construction}
\label{S:MODELCONSTRUCTION}

%Once the model has been configured, the next step is to build the model.
Model construction builds the full model matrices $A$, $C$, $Q$, $R$ by assembling the sub-matrices associated with each block component and each time-series.
The corresponding values for the model parameters are also considered during model construction.
%Because the model matrices are often time-dependent, OpenBDLM creates a function for each matrix, that depends on the time $t$, and model parameters $\bm\theta$.

\subsubsection{Model construction functions}

The model construction workflow is presented in Figure~\ref{FIG:ModelConstructionWorkflow}. The function in OpenBDLM used for model construction is:

\begin{description}[style=unboxed]
\item[Builds the model ] \leavevmode
  \begin{lstlisting}[ basicstyle = \mlttfamily \small, breaklines=true]
  [model, misc]=buildModel(data, model, misc)
 \end{lstlisting}
\end{description}

\begin{figure}[!h]
  \centering
  \captionsetup{justification=centering}
\scalebox{0.8}{
\begin{tikzpicture}


\node[parabisque](inputModelConstruct){\begin{tabular}{c} \lstinline[basicstyle = \mlttfamily \small ]!model.components.block! \\ \lstinline[basicstyle = \mlttfamily \small ]!model.components.ic! \\  \lstinline[basicstyle = \mlttfamily \small ]!model.components.const!  \\ \lstinline[basicstyle = \mlttfamily \small ]!data!  \end{tabular}};
\node[esbisque](ModelConfiguration)[below of = inputModelConstruct, yshift = -1.5cm]{\phantom{} ModelConfiguration.m \phantom{}};
\node[esbisque](defineModel)[below of = ModelConfiguration, yshift = -1cm]{\phantom{} defineModel.m \phantom{}};
\node[esbisque](buildModel)[below of = defineModel , yshift=-1cm, xshift = 0cm]{\phantom{} buildModel.m \phantom{}};

\node[parabisque](outputModelConstruct)[below of = defineModel, yshift = -3.5cm]{\begin{tabular}{c}  \lstinline[basicstyle = \mlttfamily \small ]!model.A,C,Q,R,Z!  \\ \lstinline[basicstyle = \mlttfamily \small ]!model.param_properties! \\ \lstinline[basicstyle = \mlttfamily \small ]!model.initX! \\  \lstinline[basicstyle = \mlttfamily \small ]!model.initV!  \\ \lstinline[basicstyle = \mlttfamily \small ]!model.initS!  \end{tabular}};

\path[->, thick] (inputModelConstruct)edge(ModelConfiguration);
\path[->, thick] (ModelConfiguration)edge(defineModel);
\path[->, thick] (defineModel)edge (buildModel);
\path[->, thick] (buildModel)edge (outputModelConstruct);

\end{tikzpicture} } 
\caption{Model construction workflow} \label{FIG:ModelConstructionWorkflow}
\end{figure}