\section{FAQ Troubleshooting}
%
%\begin{itemize}
%\item How to cite OpenBDLM ?\\
%
%\noindent \emph{OpenBDLM, an Open-Source Software for Structural Health Monitoring using Bayesian Dynamic Linear Models}\\{\small
%            Gaudot, I., Nguyen, L.H., Khazaeli S.and Goulet, J.-A.\\
%            Submitted to 13th International Conference on Applications of Statistics and Probability in Civil Engineering, Vol. X, Issue X, 2019\\}
%      [] [~]  [~] [] \cite{Gaudot2019OpenBDLM}\\[4pt]
%
%\end{itemize}

%
%\Que{Here first question would come}
%\Ans{\lipsum[1]\lipsum[1]}
%
%\Que{Here second question would come}
%\Ans{\lipsum[1]\lipsum[1]}


\begin{description}[style=unboxed]

\item[\textbf{How to cite OpenBDLM ?}] \leavevmode \\

\noindent \emph{OpenBDLM, an Open-Source Software for Structural Health Monitoring using Bayesian Dynamic Linear Models}\\{\small
            Gaudot, I., Nguyen, L.H., Khazaeli S.and Goulet, J.-A.\\
            Submitted to 13th International Conference on Applications of Statistics and Probability in Civil Engineering, Vol. X, Issue X, 2019\\}
      [~] [~]  [~] [~] \cite{Gaudot2019OpenBDLM}\\[4pt]


\item[\textbf{How to choose the right model structure  for my data ?}] \leavevmode \\

There is no magic: inspect the data to propose candidates of model.
Comparing the log-likelihood values obtained using different model candidates is a good indicator.
Moreover, the presence of a weird behavior in the autoregressive hidden states (trend, periodicity) may indicate that the model is incorrect or, at least, incomplete (but this may also due to model parameters values).
Note that model structure selection is a large field of study and many methods are available in the literature.


\end{description}




\noindent \todo{To be completed}