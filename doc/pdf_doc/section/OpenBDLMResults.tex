\section{Results}

\subsection{Visualizing results}
\label{S:VisualizingResults}

The figures that popup on screen aim to represent the data, the data summary, and the hidden states results for each step of the analysis.
Note that in contrast with the data plot, the data summary plot offers a way  to visualize the amplitude, the timesteps and the data availability in a compact way.
By default, a solid line connects two successive real-valued measurement, whatever the timestep.
Missing data (NaN) are not represented in the plot of the observed data, thus resulting in gap for large period  of time with missing data.
The data availability in the data summary plot indicates each missing data with a red cross, thus making it useful to detect sparse missing data which are invisible in the data amplitude plots.
The working period of the sensor is represented by a thick green line in the data availability plot.
The plot appearance may be controlled from the dedicated \lstinline[basicstyle = \mlttfamily \small]!misc.options!.

\subsubsection{Generating figures at any time}
The option \colorbox{light-gray}{\lstinline[basicstyle = \mlttfamily \small]!14!} from the main menu allows generating the different type of figure at any time (see~\ref{LST:PlotMenu}).

\subsubsection{Saving figures}
It is not advised to save figures ``manually'' using the Matlab.
It would most likely not save figures as seen on the screen.
Instead, use the OpenBDLM export facilities described in the Section~\ref{SS:ExportResults}, or set the \lstinline[basicstyle = \mlttfamily \small]!misc.options.isExportTEX!, \lstinline[basicstyle = \mlttfamily \small]!misc.options.isExportPDF!, \lstinline[basicstyle = \mlttfamily \small]!misc.options.isExportPDF! to \lstinline[basicstyle = \mlttfamily \small]!true! to automatically save the figure in a specific format each time a figure is created \footnote{Automatic figure saving is not recommanded because it is computationally expensive.}.
The workflow for visualization is shown in Figure~\ref{FIG:ResultsVisualizationsWorkflow}. 
The functions used to visualize the data and results are:

\begin{lstlisting}[ frame = single, basicstyle = \mlttfamily \small, caption = {OpenBDLM plot menu}, label = LST:PlotMenu ,  float =ht, linewidth=\linewidth, captionpos=b]
----------------------------
/    Plot
----------------------------

     1 ->  Plot data 
     2 ->  Plot data summary 
     3 ->  Plot hidden states 

     Type R to return to the previous menu

     choice >> 
\end{lstlisting}



\begin{description}[style=unboxed]
\item[Pilote function to plot data and estimations] \leavevmode
  \begin{lstlisting}[ basicstyle = \mlttfamily \small, breaklines=true]
  [misc] = pilotePlot(data,model,estimation,misc)
 \end{lstlisting}

\item[Plot data amplitude values and data timestep] \leavevmode
  \begin{lstlisting}[ basicstyle = \mlttfamily \small, breaklines=true]
[FigureNames] = plotData(data,misc,varargin)
 \end{lstlisting}

\item[Plot data amplitude, data time step, and data availability ]  \leavevmode
  \begin{lstlisting}[ basicstyle = \mlttfamily \small, breaklines=true]
  plotDataSummary(data, misc, varargin)
 \end{lstlisting}

 \item[Plot hidden states, predicted data, and model probability]  \leavevmode
  \begin{lstlisting}[ basicstyle = \mlttfamily \small, breaklines=true]
plotEstimations(data,model,estimation,misc,varargin)
 \end{lstlisting}
 
  \item[Plot true and estimated hidden states]  \leavevmode
  \begin{lstlisting}[ basicstyle = \mlttfamily \small, breaklines=true]
  [FigureNames] = plotHiddenStates(data,model,estimation,misc,varargin)
 \end{lstlisting}
 
   \item[Plot observed and predicted data]  \leavevmode
  \begin{lstlisting}[ basicstyle = \mlttfamily \small, breaklines=true]
  [FigureNames] = plotPredictedData(data,model,estimation,misc,varargin)
 \end{lstlisting}
 
    \item[Plot true and estimated model probability]  \leavevmode
  \begin{lstlisting}[ basicstyle = \mlttfamily \small, breaklines=true]
  [FigureNames] = plotModelProbability(data,model,estimation,misc,varargin)
 \end{lstlisting}
 
     \item[Waterfall plot for kernel regression component]  \leavevmode
  \begin{lstlisting}[ basicstyle = \mlttfamily \small, breaklines=true]
[FigureNames] = plotWaterfallKRegression(data,model,estimation,misc,varargin)
 \end{lstlisting}
 
 \item[Export the current figure in \LaTeX{} (tikz) file using matlab2tikz]  \leavevmode
  \begin{lstlisting}[ basicstyle = \mlttfamily \small, breaklines=true]
 exportPlot(FigureName,varargin)
 \end{lstlisting}
 
\end{description}


\begin{figure}[!h]
  \centering
  \captionsetup{justification=centering}
\scalebox{0.8}{
\begin{tikzpicture}

\node[parawhite](inputResultsVisualization){\begin{tabular}{c} \lstinline[basicstyle = \mlttfamily \small ]!data!  \\ \lstinline[basicstyle = \mlttfamily \small ]!estimation! \\ \lstinline[basicstyle = \mlttfamily \small ]!misc!  \end{tabular}};
\node[eswhite](pilote)[below of = inputResultsVisualization, yshift=-1cm, xshift = 0cm]{ \phantom{} pilotePlot.m \phantom{} };
\node[eswhite](plotData)[below of = pilote, yshift=-1cm, xshift = 0cm]{ \phantom{} plotData.m \phantom{} };
\node[eswhite](plotDataSummary)[below of = plotData, xshift = 0cm, yshift=-1cm, ]{ \phantom{} plotDataSummary.m \phantom{} };
\node[eswhite](plotEstimations)[below of = plotDataSummary, yshift= -0.8cm]{\phantom{} plotEstimations.m \phantom{}};
\node[eswhite](plotAll)[right of = plotEstimations, yshift= 0cm, xshift = 4cm]{\begin{tabular}{c} plotHiddenStates.m \\ plotPredictedData.m \\ plotModelProbability.m \\ plotWaterfallKRegression.m  \end{tabular}};
\node[testwhite](testExport)[below of = plotEstimations, yshift= -1.5cm, xshift = 0cm]{\begin{tabular}{c} export \\ figures ?  \end{tabular}};
\node[eswhite](Export)[right of = testExport , yshift=0cm, xshift = 2.5cm]{ \phantom{}  exportPlot.m \phantom{} };
\node[parawhite](outputExport)[right of = Export , yshift=0cm, xshift = 2.5cm]{ \begin{tabular}{c} PNG, PDF, or \\ \LaTeX{} figures \end{tabular}};
\node[parawhite](outputResultsVisualization1)[below of =testExport, yshift=-2cm, xshift = 0cm]{\begin{tabular}{c}  \MATLAB{} figure \\ on screen  \end{tabular}};

\path[->, thick] (inputResultsVisualization)edge(pilote);
\path[->, thick] (pilote)edge(plotData);
\path[->, thick] (plotData)edge(plotDataSummary);
\path[->, thick] (plotDataSummary)edge(plotEstimations);
\path[->, thick] (plotEstimations)edge(plotAll);
\path[->, thick] (plotEstimations)edge(testExport);
\path[-, thick] (testExport)edge  node[anchor=center, above, rotate=0, rotate=0]{ yes}  (Export);
\path[-, thick] (Export)edge(outputExport);
\path[->, thick] (testExport)edge node[anchor=center, above, rotate=0, rotate=90]{ no}  (outputResultsVisualization1);
\path[->, draw,  thick] (outputExport.east) -| (9cm, -12.95cm) |- (outputResultsVisualization1.east);

\end{tikzpicture} } 
\caption{Visualization results workflow} \label{FIG:ResultsVisualizationsWorkflow}
\end{figure}


\subsection{Exploring results}
\label{S:ExploringResults}

Exploring results (raw data and figures), is essential for interpretation, validation and reporting during and after the analysis.
During the analysis, the \MATLAB{} binary RES\_*.mat, PROJ\_*.mat and DATA\_*.mat files are automatically created for this purpose.
Moreover, OpenBDLM enables exporting results in user's specified format using the export menu (option \colorbox{light-gray}{\lstinline[basicstyle = \mlttfamily \small]!17!} from the main menu).
For instance, the results can be exported in CSV format for direct use in a third party software.
The figures can be exported in PDF, PNG, and \LaTeX{} (tikz) to create publication-quality figures.

\subsubsection{RES\_*.mat result file}

The \lstinline[basicstyle = \mlttfamily \small]!RES_*.mat! results file are located in the ``results/mat'' subfolder.
The \MATLAB{} binary .MAT contain four \MATLAB{} variables called  \lstinline[basicstyle = \mlttfamily \small]!timestamps!, \lstinline[basicstyle = \mlttfamily \small]!Mean!, \lstinline[basicstyle = \mlttfamily \small]!StandardDeviation!, and  \lstinline[basicstyle = \mlttfamily \small]!labels!.
\begin{itemize}
\item \lstinline[basicstyle = \mlttfamily \small]!timestamps!: $\mathtt{N}\times 1$ array containing the time vector.
\item \lstinline[basicstyle = \mlttfamily \small]!Mean!: $\mathtt{N}\times (\mathtt{L}+\mathtt{D}+1)$ array containing the filtered or smoothed posterior mean values of the hidden states, where $\mathtt{L}$ is the number of hidden states,  $\mathtt{D}$ is the number of time series.
\item \lstinline[basicstyle = \mlttfamily \small]!StandardDeviation!: $\mathtt{N}\times (\mathtt{L}+\mathtt{D}+1)$ array containing the filtered or smoothed posterior standard deviation values of the hidden states.
\item \lstinline[basicstyle = \mlttfamily \small]!labels!: $1\times (\mathtt{L}+\mathtt{D}+1)$ cell array containing the label of each column of the \lstinline[basicstyle = \mlttfamily \small]!Mean! and \lstinline[basicstyle = \mlttfamily \small]!StandardDeviation! arrays.
\end{itemize}

The function used to save the result is:

\begin{description}[style=unboxed]
\item[Save results in a .mat file] \leavevmode
  \begin{lstlisting}[ basicstyle = \mlttfamily \small, breaklines=true]
[misc]=saveResultsMAT(data,model,estimation,misc,varargin)
 \end{lstlisting}
\end{description}

\subsubsection{PROJ\_*.mat project file}

The \lstinline[basicstyle = \mlttfamily \small]!PROJ_*.mat! project file are located in the ``saved\_projects/mat'' subfolder.
This files  contains the internal variables \lstinline[basicstyle = \mlttfamily \small ]!model!,\lstinline[basicstyle = \mlttfamily \small ]!estimation!, \lstinline[basicstyle = \mlttfamily \small ]!misc!.
The  content of those internal variables is described in Section~\ref{S:OpenBDLMINPUTOUTPUT}.
The function used to save the project is:
\begin{description}[style=unboxed]
\item[Save the variables model, estimation, misc in a .mat project file] \leavevmode
  \begin{lstlisting}[ basicstyle = \mlttfamily \small, breaklines=true]
saveProject(model, estimation,misc,varargin)
 \end{lstlisting}
\end{description}

\subsubsection{DATA\_*.mat data file}

The \lstinline[basicstyle = \mlttfamily \small]!DATA_*.mat! data file are located in the ``data/mat'' subfolder.
This file contains three variables called \lstinline[basicstyle = \mlttfamily \small]!labels!, \lstinline[basicstyle = \mlttfamily \small]!timestamps!, and \lstinline[basicstyle = \mlttfamily \small]!values! that fully describe the time series data.
The content of those variables is described in Section~\ref{SS:MATInput}.
The function used to save the data is:
\begin{description}[style=unboxed]
\item[Save data in a binary Matlab .mat file] \leavevmode
  \begin{lstlisting}[ basicstyle = \mlttfamily \small, breaklines=true]
[misc, dataFilename] = saveDataBinary(data,misc,varargin)
 \end{lstlisting}
\end{description}

\subsubsection{Exporting results}
\label{SS:ExportResults}

The option \colorbox{light-gray}{\lstinline[basicstyle = \mlttfamily \small]!17!} from the main menu offers a way to export the data, the results, the project and the figures in specific format (see~\ref{LST:ExportMenu}).
It is possible to export figures in PDF, PNG\footnote{Yair Altman, 2011, export\_fig, \url{https://www.mathworks.com/matlabcentral/fileexchange/23629-export\_fig}} and \LaTeX{}\footnote{Nico Schl�mer,2013, matlab2tikz, \url{https://www.mathworks.com/matlabcentral/fileexchange/22022-matlab2tikz-matlab2tikz}} \footnote{All the time-series figures shown in this document have been created from \LaTeX{} (tikz) files output from OpenBDLM. Minor post-processing have been done on the \LaTeX{} file. Compilation using LuaLatex.}.
It is possible to export the results and the data in .CSV files.
OpenBDLM creates one three-columns CSV file for each posterior filtered or smoothed hidden states for each time series, as well as CSV files for the predicted data for each time series, and a CSV file for the model probability.
The first column gives the timestamp, the second column the mean, and the third column the standard deviation.
The first line of each file is the header, that gives the reference name of the time-series associated with the hidden states as well as the date of the first timestamp in the \textquotesingle YYYY-DD-MM-HH-MM-SS\textquotesingle{} format. 
It is also possible to export the project in a configuration file that respects the format described in Section~\ref{S:CFGFile}.

\begin{lstlisting}[ frame = single, basicstyle = \mlttfamily \small, caption = {OpenBDLM export menu}, label = LST:ExportMenu ,  float =ht, linewidth=\linewidth, captionpos=b]
----------------------------------
/    Export
----------------------------------

     1 ->  Export the project in a configuration file
     2 ->  Export data in CSV format
     3 ->  Export results in CSV format
     4 ->  Create and export figures 

     Type R to return to the previous menu

     choice >> 
\end{lstlisting}

The export workflow is shown in Figure~\ref{FIG:ExportWorkflow}, and the functions used to export the results are:
\begin{description}[style=unboxed]
\item[Pilote function to export data, estimations and project] \leavevmode
  \begin{lstlisting}[ basicstyle = \mlttfamily \small, breaklines=true]
[misc]=piloteExport(data,model,estimation,misc)
 \end{lstlisting}

\item[Create and print a configuration file from project] \leavevmode
  \begin{lstlisting}[ basicstyle = \mlttfamily \small, breaklines=true]
  [configFilename] = printConfigurationFile(data,model,estimation,misc,varargin)
 \end{lstlisting}

\item[ Save time series data in separate .csv files] \leavevmode
  \begin{lstlisting}[ basicstyle = \mlttfamily \small, breaklines=true]
[misc] = saveDataCSV(data,misc,varargin)
 \end{lstlisting}

\item[ Save results in .CSV files] \leavevmode
  \begin{lstlisting}[ basicstyle = \mlttfamily \small, breaklines=true]
[misc]=saveResultsCSV(data, model, estimation, misc, varargin)
 \end{lstlisting}

\end{description}


\begin{figure}[!h]
  \centering
  \captionsetup{justification=centering}
\scalebox{0.8}{
\begin{tikzpicture}

\node[parawhite](inputExport){\begin{tabular}{c} \lstinline[basicstyle = \mlttfamily \small ]!data!  \\ \lstinline[basicstyle = \mlttfamily \small ]!estimation! \\ \lstinline[basicstyle = \mlttfamily \small ]!misc!  \end{tabular}};
\node[eswhite](pilote)[below of = inputExport, yshift=-1cm, xshift = -0cm]{ \phantom{} piloteExport.m \phantom{} };
\node[eswhite](CFG)[below of = pilote, yshift=-1cm, xshift = -5.5cm]{\begin{tabular}{c} printConfiguration \\ File.m \end{tabular}};
\node[eswhite](DataCSV)[below of = pilote, yshift=-1cm, xshift = -1.75cm]{\begin{tabular}{c} saveData\\CSV.m   \end{tabular}};
\node[eswhite](ResultsCSV)[below of = pilote, yshift=-1cm, xshift = 1.5cm]{\begin{tabular}{c} saveResults\\CSV.m  \end{tabular} };
\node[eswhite](ExportFig)[below of = pilote, yshift=-1cm, xshift = 5.15cm]{ \phantom{} see Figure~\ref{FIG:ResultsVisualizationsWorkflow} \phantom{} };
\node[parawhite](outputCFG)[below of = CFG , yshift=-1cm, xshift = 0cm]{\begin{tabular}{c}  CFG\_*.m \\ file \end{tabular}};
\node[parawhite](outputDCSV)[below of = DataCSV , yshift=-1cm, xshift = 0cm]{\begin{tabular}{c} .CSV data \\ files \end{tabular}};
\node[parawhite](outputRCSV)[below of = ResultsCSV , yshift=-1cm, xshift = 0cm]{ \begin{tabular}{c} .CSV results \\ files \end{tabular}};
\node[parawhite](outputExport)[below of = ExportFig , yshift=-1cm, xshift = 0cm]{ \begin{tabular}{c} PNG, PDF, or \\ \LaTeX{} figures \end{tabular}};

\path[->, thick] (inputExport)edge(pilote);
\path[->, thick] (CFG)edge(outputCFG);
\path[->, thick] (DataCSV)edge(outputDCSV);
\path[->, thick] (ResultsCSV)edge(outputRCSV);
\path[->, thick] (ExportFig)edge(outputExport);
\path[->, draw,  thick] (pilote.south) |- (-4cm, -3cm) -| (CFG.north);
\path[->, draw,  thick] (pilote.south) |- (-2cm, -3cm) -| (DataCSV.north);
\path[->, draw,  thick] (pilote.south) |- (3cm, -3cm) -| (ResultsCSV.north);
\path[->, draw,  thick] (pilote.south) |- (5cm, -3cm) -| (ExportFig.north);

\end{tikzpicture} } 
\caption{Export options workflow} \label{FIG:ExportWorkflow}
\end{figure}







