\documentclass{book}


\usepackage{hyperref}
\hypersetup{colorlinks=true,linkcolor=blue}

\title{OpenBanana: An open-source Matlab software for Structural Health Monitoring using Bayesian dynamic linear models}
\author{Ianis Gaudot, Luong H. Nguyen, James-A. Goulet \\ Polytechnique Montreal}

\begin{document}
\maketitle

\tableofcontents

\chapter{What OpenBanana does ?}
\newpage

\chapter{Install OpenBanana}
\newpage

\chapter{Data processing}
\section{Input data format}
\section{Output data format}
\section{Merging timestamps vector}
\section{Associated functions}
\newpage

\chapter{Model building}
\section{Purpose}
\section{Model class}
\section{Dependencies}
\subsection{Observed covariate}
\subsection{Hidden covariate}
\section{Model components}
\subsection{Local level}
\subsection{Local trend}
\subsection{Local acceleration}
\subsection{Local level compatible trend}
\subsection{Local level compatible acceleration}
\subsection{Local trend compatible acceleration}
\subsection{Periodic component}
\subsection{First order autoregressive component}
\subsection{Dynamic regression component}
\subsection{Static kernel regression component}
\subsection{Dynamic kernel regression component}
\section{Associated functions}
\newpage

\chapter{Model parameters learning}
\section{Purpose}
\section{Posterior}
\subsection{Prior}
\subsection{Likelihood}
\section{Model parameters bounds and transformed spaces}
\subsection{Logarithmic transformation}
\subsection{Sigmoid transformation}
\section{Gradient-based optimization techniques}
\subsection{Newton-Raphson approach}
\subsection{Stochastic Gradient Ascent approach}
\section{Constrain model parameters between each others}
\section{Associated functions}
\newpage

\chapter{Hidden states estimation}
\section{Kalman equations}
\section{UD computation}
\section{Filtering}
\section{Smoothing}
\section{Associated functions}
\newpage

\chapter{Data simulation}
\section{Purpose}
\section{Associated functions}
\newpage

\chapter{Model validation}
\section{Prediction capacity}
\section{Posterior model parameters covariance matrix analysis}
\section{Posterior state covariance matrix analysis}
\section{Residual component analysis}
\section{Associated functions}
\newpage

\chapter{Visualization tools}
\section{Purpose}
\section{Data availability plots}
\section{Hidden states plots}
\section{Export figures in PDF, PNG, TEX}
\section{Associated functions}
\newpage


\chapter{Version control}
\section{Purpose}
\section{Associated functions}
\newpage

\chapter{Examples}
\section{Example 1}
Simulated data with: one time series, one model class, \{[12 31 41]\}.

\section{Example 2}
Simulated data with: two time series with dependencies, one model class, \{[11 41], [11 31 41]\}.

\section{Example 3}
Simulated data with one time series, two model classes, \{[21 31 41]\} and \{[12 31 41]\}.

\section{Example 4}
Real data with one time series, one model class, \{[12 52 41]\}.

\newpage

\chapter{List of functions}
\newpage


\chapter{Older versions}
\newpage

\chapter{Last changes}

This new version now includes the data loader and the data simulation tools.
It also includes additionnal tools to export the figures in several format (.tex, .png, .pdf).
This version also supports batch processing (loop over input files, see the script demo\_1.m). \\

The model parameters and hidden state estimation functions are (almost) identical compared to the previous GitHub version.\\

WARNING: the 'data' structure completely changed (cell array instead of simple array) then previous configuration file are NOT readable with this version. You have to initialize a new project using the interactive tool, and then copy/paste manually your previously learned initial hidden states and/or model parameters values.\\

Note that the sofware will only work properly if you run it from the same folder that contains the file BDLM.m.
As usual, always put all the folders in your Matlab path before running the program.

\section{Data loading}
Data loader is accessible via interactive tool (option '0').
See the DataLoader.m function to change data loading option (handle quantity of allowed missing data, handle tolerance between timestamp, etc...).

\section{Data simulation}
Once a project has been created, data simulation is now possible through option '16'.
A new project is created.

\section{Create a configuration file}
Option '0' builds a configuration file by default.
At any time, option '17' export current model in configuration file format.

\section{Export figures}
The program can now save figures in tex, png and pdf format.
See the plotEstimation.m function to change export options.
When pdf export is done, the program builds a single pdf file containing all the figures.

\section{Batch processing}

Batch processing allows you to plan a succession of tasks in advance, and let them run during the night (or whenever you want).
All the planned command have to be written in ASCII input files.\\

See demo\_1.m for an example of batch processing. demo\_1.m sequentially creates 3 simulated dataset.\\

WARNING: the beta version of this tool uses global variables (named 'isAnswersFromFile' 'AnswersFromFile' 'AnswersIndex'), so always clean your workspace by typing 'clear -global isAnswersFromFile AnswersFromFile AnswersIndex' before and after using batch processing.

\newpage


\end{document}

